\chapter{Introdução}
\label{cap:introducao}

Coloque aqui o texto da introdução, contextualizando o seu trabalho...

Testando o uso das siglas na - pela primeira vez para \gls{ACM}. Segunda vez para \gls{ACM}...

blabla... \gls{UTFPR}

xxx \gls{TCP}

\acrlong{UTFPR}

A rede \gls{IP}...

% Sugestões de seções
\section{Considerações preliminares}

Aqui você pode descrever problemas, soluções e outros assuntos que ajudem a introduzir o leitor ao contexto de seu trabalho...

\section{Problema de Pesquisa}
\label{cap:introducao:sec:problema:pesquisa}

Descreva o seu Problema de Pesquisa com as questões de pesquisa que seu trabalho irá tentar responder.

\begin{itemize}
    \item How often do breaking changes manifest in the client package?
\end{itemize}
Currently, the npm packages contain most of 11 billion weekly downloads and most of 45 billions of downloads a month. npm packages are used by millions of peoples and projects around the world. They are free, small and easy to use. However, the simple release that contains an error can affect a lot of other packages because the many projects depend on each other directly or indirectly, and when one introduces a breaking change, many and many are affected. However, an unspoken release that contains some error may cause the same loss - and worse. To avoid this, npm use ranges based in \textit{SEMVER} to prevent these errors from affecting the packages. Prior research shows that the distinction between breaking and non-breaking changes isn’t always clear, and it may be difficult for the library developer to decide how to increment version numbers for a new release \cite{noregrets2018}. Therefore, understanding how often provider packages publish breaking changes can help client packages to make better decisions about how and when to update a provider version.

To answers this research question, we’ve focused on three points: 1) how many times each provider package publish a release containing breaking change - and what is the percentage of releases with breaking change; 2) in which release level breaking changes are typically introduced based in \textit{SEMVER} - patch, minor or major; 3) what is the percentage of the clients of a provider that updates towards a release with breaking change. To analyze all points, the sorted packages were clone from \textit{GitHub}, updated the index and the files in the working tree using the command \textit{git checkout} with a timestamp of release, excluded the \textit{package-lock.json} and executed \textit{npm install} and \textit{npm test}. The result of this execution -- successful or error -- was saved with information about client version and \textit{Node.js} version that was executed and the version based in the timestamp of release. Then, each breaked package was analyzed manually to discovery if the break was a breaking change.

\begin{itemize}
    \item What issues in the provider package cause the manifestation of a breaking change?
\end{itemize}

\begin{itemize}
    \item How do client packages recover from the manifestation of breaking change?
\end{itemize}

Break changes appear in few times. But, once a breaking change appears, the provider should fix. It’s necessary because, in the npm ecosystem, where hundreds of thousands of packages are connected, a single wrong release can break a lot of other packages. An example of this is the package called \textit{left-pad}, which was removed from npm by your developer and break many thousands of projects on 2.5 hours.

The responsibility to fix the release is from the provider, but, sometimes, the provider cannot fix the release/package for some reason. Then, the customer gets this responsibility because your wrong provider affects it and all of your customers are also affected. In the example of \textit{left-pad}, the provider never published a new release and, for all packages to continue working, the client's packages that have fixed this error publishing a new package with the same name and the same code.

However, the client may not know how to solve the problem. If the client is a simple user that only knows a little bit about the internal works from the provider, he can hardly solve in its own code an error that is being caused by the provider. This is very bad for the clients because he should fix the error, change to a previous version or choose another provider.

To answer the final research question, we analyze three points: 1) What happened so the provider couldn’t fix the code; 2) How the client fix the error or just notified the provider; 3) How many times the client fixed the error. All information about this research question is on the \textit{release-notes, issues}, and \textit{pull-requests}.

\section{Objetivos}
\label{cap:introducao:sec:objetivos}

Descreva de maneira sucinta os objetivos de seu trabalho (o que você fará durante o desenvolvimento de TCC 1 e TCC 2?). Faça um texto BEM curto e objetivo...

\section{Contribuições}
\label{cap:introducao:sec:contribuicoes}

No que o seu trabalho ajuda? Há diferenças entre o seu trabalho e outros?

\section{Organização do Texto}
\label{cap:introducao:sec:organizacao:texto}

No Capítulo~\ref{cap:introducao} blablabla, no capítulo seguinte tititi, etc... Nossa proposta é apresentada no Capítulo~\ref{cap:proposta}.... Finalmente, no Capítulo~\ref{cap:conclusoes} apresentamos as conclusões obtidas no desenvolvimento deste trabalho...