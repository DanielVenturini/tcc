\chapter{Introdução}
\label{cap:introducao}

O \textit{Node Package Manager} (\textsf{npm}) é um gerenciador de pacotes para a linguagem \textsf{JavaScript} e um registro no qual os pacotes são publicados e armazenados. Lançado em 2009, seu principal objetivo é facilitar o compartilhamento de código escrito para o \textsf{Node.js} e, atualmente, o registro do \textsf{npm} ocupa a posição de maior repositório para uma dada linguagem, com mais de 1,4 milhões de pacotes.\footnote{http://www.modulecounts.com} O \textsf{npm} é um dos fatores que impulsionaram o \textsf{JavaScript} a se tornar um ecossistema completo, com pacotes, \textit{frameworks}, aplicativos \textit{web} entre outros \cite{introduction:npm}. Além disso, 97\% dos aplicativos \textit{web} são oriundos do \textsf{npm}, de acordo com o próprio \textsf{npm}.\footnote{https://blog.npmjs.org/post/180868064080}

O \textsf{npm} contém o maior número de dependências entre os diversos gerenciadores de pacotes para uma linguagem de programação \cite{teorical_reference:npm_2}. Nesse cenário, o termo \textit{provedor} refere-se ao pacote que provê recursos para os seus \textit{clientes}, que usufruem desses recursos, conforme definidos na Seção \ref{ref-teo:prov_clie}. Como há muitas dependências no \textsf{npm}, quando algum provedor contém algum tipo de defeito, um grande número de clientes podem ser afetados. Para dimensionarmos a cadeia de dependências que os pacotes no \textsf{npm} têm, considere o seguinte exemplo real. O pacote \textsf{left-pad} foi removido do \textsf{npm} e essa remoção impactou milhares de outros pacotes em apenas 2,5 horas, gerando um erro quando o comando \texttt{npm install} era executado nesses pacotes. Até mesmo pacotes renomados como o \textsf{babel} e o \textsf{atom} foram atingidos, e por sua vez propagaram esse defeito para seus clientes. Assim, problemas entre provedores e clientes realmente ocorrem no ecossistema do \textsf{npm} devido a essa cadeia de dependências entre os pacotes, e por isso esse ecossistema foi escolhido como estudo de caso.

O \textsf{npm} utiliza um sistema de versionamento chamado de \textit{Versionamento Semântico}, conforme definido na Seção \ref{ref-teo:semver}, que é composto basicamente de três dígitos (níveis) separados por um ponto (\textit{.}) e que devem ser incrementados de acordo com as alterações que o desenvolvedor introduz na \textit{release}. Esse sistema de versionamento permite que o provedor publique \textit{releases} com \textit{breaking changes}, novas funcionalidades e correções de erros, respectivamente nos níveis \textit{major}, \textit{minor} e \textit{patch}, separadas uma das outras. As \textit{breaking changes}, descritas na Seção \ref{ref-teo:breaking_change}, são alterações nos provedores que os tornam incompatíveis com as suas versões anteriores \cite{intro:break_change}, fazendo com que os provedores tenham um comportamento inesperado para os clientes. Os provedores podem introduzir \textit{breaking changes} para evoluir seus códigos, fazer alterações importantes entre outros motivos. Essas \textit{breaking changes} devem ser introduzidas apenas em \textit{releases} versionada com a incrementação do nível \textit{major} do Versionamento Semântico. Porém, o problema está quando uma \textit{breaking change} é introduzida em uma \textit{release} do nível \textit{minor} ou \textit{patch}, ou seja, \textit{releases} que deveriam apenas conter novas funcionalidades e correções de erros, respectivamente. Quando isso ocorre, erros são introduzidos nos clientes, que são impactados por \textit{breaking changes}.

Esse trabalho tem por objetivo realizar um estudo empírico sobre as \textit{breaking changes} que foram introduzidas erroneamente em níveis \textit{minor} e \textit{patch} no ecossistema do \textsf{npm}. Além de quantificar as \textit{breaking changes}, este trabalho apresenta uma categorização das \textit{breaking changes} e uma análise acerca de como os clientes se recuperam das \textit{breaking changes}. Para isso, foi utilizada uma amostra representativa dos clientes no \textsf{npm} e, para cada uma de suas \textit{releases}, foi resolvida a versão de todos os provedores para verificar se havia alteração nas versões que os clientes aceitavam dos provedores. Ou seja, para cada provedor em uma determinada \textit{release}, foi verificado qual era a última versão disponível do provedor até a data da \textit{release} do cliente. Então, determinou-se a versão de cada provedor que o cliente utilizava quando publicou sua \textit{release}. Posteriormente, o repositório do cliente foi clonado e foi executado os comandos \texttt{npm install} e \texttt{npm test} para instalar os provedores e executar os testes de cada \textit{release} do cliente. Após, foi realizada uma analise manual em cada \textit{release} que resultou em erro para confirmar se o erro se tratava de uma \textit{breaking change} ou não, ou seja, se o erro teve origem no provedor ou no cliente, respectivamente. Por fim, foram analisados os repositórios dos provedores e dos clientes para recolher informações pertinentes a cada caso de \textit{breaking change}, tais como \textit{issues} e \textit{pull-requests}.

Este trabalho contribui como um estudo empírico sobre \textit{breaking changes} no ecossistema do \textsf{npm}. Os resultados e as discussões apontam os principais casos de \textit{breaking changes} e os métodos com os quais os provedores podem utilizar para mitigar a introdução de \textit{breaking changes}. Ainda, descrevemos técnicas que os clientes podem utilizar para gerenciar seus provedores, descobrirem e recuperarem-se com eficiência das \textit{breaking changes}.

O Capítulo \ref{cap:ref-teorico} contém o referencial teórico deste trabalho, juntamente com os trabalhos relacionados. No Capítulo \ref{cap:exemplos} estão descritos três exemplos reais de \textit{breaking changes} que foram identificados em nosso estudo. O Capítulo \ref{cap:qp} contém a motivação e o método para cada uma das questões de pesquisa. O Capítulo \ref{cap:metodologia} descreve a coleta dos dados que foram utilizados nessa pesquisa. Enquanto o Capítulo \ref{cap:results} apresenta os resultados das questões de pesquisa, o Capítulo \ref{cap:discussoes} discute esses resultados. No Capítulo \ref{cap:threats} há a descrição dos fatores de ameças à validade interna, externa e de construção deste trabalho. Por fim, o Capítulo \ref{cap:conclusoes} apresenta as conclusões deste trabalho.