\chapter{Draft}

\section{RQ1. Com que frequência \textit{breaking changes} surgem \filipe{afetam? impactam?} nos pacotes clientes?}
\subsection{Método}
\label{apr:rq1}

%\Gls{NPM}.
\filipe{O stack trace veio meio do nada ... começe explicando que você identificou breaking changes a partir de uma análise do stack trace ... aí começer a falar o que é isso. Acho que essa informação (como breaking changes foram identificadas) deve ficar na seção de coleta de dados, pois é transversal a todas as RQs.}
Um \textit{stack trace} é utilizado pelo \gls{NPM} para apresentar informações sobre um determinado erro. Quando os comandos \textit{npm install} e \textit{npm test} resultam em erro, o \Gls{NPM} mostra o erro e todas as chamadas de funções, incluindo as invocações para os provedores. A Figura \ref{fig:trace} mostra um exemplo genérico de um \textit{stack trace} exibido pelo \Gls{NPM}. Nessa Figura, no topo do \textit{stack trace}, contém o tipo do erro que interrompeu a execução \filipe{do pacote client} e a sua mensagem. Nas linhas abaixo, há todas as funções e arquivos que foram executados até a manifestação do erro. Com todos estes dados, o \textit{stack trace} foi a base para o rastreamento de cada erro, uma vez que ele foi utilizado para detectar as \textit{breaking changes}, pois através do \textit{stack trace} foi possível identificar com exatidão em qual pacote o erro se manifestou: no ciente ou no provedor.\filipe{o erro sempre se manifesta no cliente, acho que o lance é que você consegue identificar se o erro foi proveniente de uma chamada a uma função do provedor ou do próprio cliente (ou algum outro provedor que não interessa à análise).}

\begin{figure}
    \centering
    \includegraphics[scale=0.7]{figuras/stack_trace.jpeg}
    \caption{\textit{stack-trace} genérico}
    \label{fig:trace}
\end{figure}{}

Para quantificar as \textit{breaking changes}, foi necessário diferenciar entre um erro que foi causado pelo próprio pacote cliente, no qual não houve influência de nenhum provedor, e um erro que foi causado por algum dos provedores, sendo assim uma \textit{breaking change}. Esta diferenciação é necessária pois um determinado erro pode ter ocorrido no código do cliente e não em um provedor, assim não sendo um caso de \textit{breaking change}. Para realizar esta diferenciação, foi utilizado \filipe{foram utilizadas} as seguintes heurísticas \filipe{para analisar o stack trace}:

\begin{itemize}
    \item Verificar no \textit{stack trace}: \filipe{remover}
    \begin{itemize}
        \item Quando não houve registro de execução dos provedores \filipe{de uma função do provedor} no \textit{stack trace}, provavelmente \filipe{acho que certamente!} o erro não foi causado por uma \textit{breaking change}. Assim, o erro podia estar apenas no código do cliente; e
        \item Quando houve registro de execução dos provedores \filipe{de uma função do provedor} no \textit{stack trace}, o erro provavelmente se tratava de uma \textit{breaking change}. Entretanto, as chamadas para \textit{frameworks} de teste, como o \textit{Mocha\footnote{https://www.npmjs.com/package/mocha}, Jasmine\footnote{https://www.npmjs.com/package/jasmine}} entre outros, ou automatizadores de tarefas, como o \textit{Grunt\footnote{https://www.npmjs.com/package/grunt}} por exemplo, não evidenciavam, inicialmente, a presença de \textit{breaking changes} uma vez que eles apenas iniciam a execução do pacote. \filipe{Não entendi essa última parte} Porém, não foi descartada a hipótese deles apresentarem \textit{breaking changes}.
    \end{itemize}{}

    \item Próximos \textit{commits} do cliente \filipe{Commits realizados pelos clientes}: foi verificado no \textit{GitHub} se o cliente tentou consertar algum erro após a \textit{release} que apresentou o erro. Se foi encontrado algum \textit{commit} com correções, foram feitas estas alterações no código do cliente para verificar se as modificações encontradas no \textit{GitHub} realmente refletiam a correção do erro. Assim, se as alterações apenas no código do cliente refletiam na correção do erro, sem que haja influência dos provedores, então o erro não se tratava de uma \textit{breaking change};

    \item Sistemas integrados ao \textit{GitHub}: alguns sistemas integrados ao \textit{GitHub} auxiliaram na investigação. Esses sistemas são o \textit{Travis\footnote{https://travis-ci.org}, Codeship\footnote{https://codeship.com}} entre outros, que armazenam os resultados da execução do pacote para cada \textit{commit}. Eles foram utilizados da seguinte maneira: se nesses sistemas integrados, a execução no \textit{commit} da \textit{release} do cliente foi realizado com sucesso e, ao executá-lo nesta pesquisa, resultou em erro, então esse caso evidencia a ocorrência de uma \textit{breaking change}, uma vez que o código do cliente estava na mesma \textit{working tree} do \textit{commit}. Mas, se a execução do cliente no momento do \textit{commit} resultou em erro, provavelmente os próximos \textit{commits} contêm alguma informação sobre o erro e sua correção, uma vez que estes sistemas integrados avisaram os desenvolvedores sobre o erro na execução.
    
    \begin{figure}
        \centering
        \includegraphics[scale=0.6]{figuras/false_positive.png}
        \caption{\textit{Script} requerido para executar com sucesso o pacote \textit{node-qrious}}
        \label{fig:false-positive}
    \end{figure}{}

    Em particular, o \textit{Travis} desempenhou um papel fundamental para identificar os erros, em especial, os falso-positivos -- casos que resultaram em erro, mas não eram. Um exemplo de falso-positivo ocorreu no pacote \textit{node-qrious}\footnote{https://www.npmjs.com/package/node-qrious}, que resultou em erro na execução, mas na análise manual, através do arquivo \textit{.travis.yml}\footnote{https://github.com/neocotic/node-qrious/blob/176ea348b9e51a8c1f0c5e2caa6cd4b0320ea5e2/.travis.yml} -- arquivo de configuração para o sistema integrado -- foi descoberto que o pacote requeria bibliotecas terceiras que, ao serem instaladas, resultou em sucesso na execução do pacote.
\end{itemize}{}

Portanto, cada erro foi analisado manualmente, com alterações no código do cliente, para certificar se o erro era um falso-positivo, um erro interno, uma \textit{non-break change} ou uma \textit{break change}. Essa separação foi importante para esta e para as próximas questões de pesquisa. Com isso, foi possível quantificar os casos \textit{breaking changes} por pacotes e por \textit{releases}.

%---------------------------------------------------%
\section{RQ2. Como os pacotes provedores introduzem \textit{breaking changes} em uma \textit{release}?}
\label{sec:rq2}

\subsection{Método}
\label{apr:rq2}
O objetivo da análise manual é descobrir o motivo que originou uma \textit{breaking changes}, ou seja, qual foi a alteração que o provedor realizou que causou a \textit{breaking change}, para que seja possível agrupa-las por suas similaridades. Porque o \textit{stack trace} sempre apresenta o erro de uma maneira genérica, às vezes, a mensagem de erro pode induzir a interpretação errônea do real motivo que originou a falha. Assim, o melhor local para se investigar quais foram as alterações que o provedor realizou é o \textit{GitHub}, no qual várias técnicas foram utilizadas para recuperar as informações necessárias:

\begin{itemize}
    \item Arquivos de alterações: os arquivos de registros de alterações, comumente nomeados por \textit{CHANGELOG.md} ou \textit{HISTORY.md}, contêm as descrições das principais alterações em cada \textit{releases} do projeto. Através da versão do provedor que foi descarregada do \gls{NPM}, foi verificado nos arquivos de alterações quais foram as modificações introduzidas pelos provedores e se alguma destas alterações diz respeito ao erro encontrado no cliente. Uma das informações mais relevantes nestes arquivos são as descrições de \textit{breaking changes}. Por exemplo, a versão \textit{5.0.0} do pacote \textit{Mocha} contém uma \textit{breaking change} que foi documentada no \textit{CHANGELOG.md}\footnote{https://github.com/mochajs/mocha/blob/master/CHANGELOG.md\#500--2018-01-17} de acordo com a Figura \ref{fig:bc_documentation} (a). Outro tipo de documentação equivalente são as \textit{releases-notes}, como pode ser visualizado na Figura \ref{fig:bc_documentation} (b) como o pacote \textit{wpxml2md} documentou \textit{breaking changes} nas \textit{releases-notes}\footnote{https://github.com/akabekobeko/npm-wpxml2md/releases/tag/v2.0.0}. Entretanto, apenas 46\% dos repositórios utilizados nesta pesquisa contêm algum dos dois registros.

    \item \textit{Issues/Pull-requests}: uma vez que uma \textit{breaking change} se manifesta em algum cliente, ele pode -- e deve -- registrar este erro através de uma \textit{issue} no repositório do provedor. O proveito de buscar informações nas \textit{issues} é que essas contêm comentários dos provedores e da comunidade, assim, há muitas informações sobre um determinado erro, além de várias outras \textit{issues} lincadas, ampliando a busca por informações. Da mesma maneira os \textit{pull-requests} foram utilizados para buscar informações sobre as \textit{breaking changes}.

    \item Versões prévias dos provedores: um ponto muito importante foi a instalação de versões prévias dos provedores. Uma vez que foi identificado qual provedor está causando a \textit{breaking change}, a instalação de outras versões ajudaram a descobrir a partir de qual \textit{release} do provedor a \textit{breaking change} foi introduzida, ou a partir de qual \textit{release} ela foi consertada. Com isso, as \textit{breaking change} ficaram mais fáceis de serem identificadas pois, uma vez que foi localizada a \textit{release} que introduziu o erro, pode ser utilizado ferramentas de \textit{diff} para analisar o código introduzido e removido daquela \textit{release}.

    \item Ferramentas de \textit{diff}: o uso da ferramenta que realizam o  \textit{diff} entre duas \textit{releases} de um pacote foi muito importante. Foi utilizado a ferramenta \textit{npm-diff}\footnote{https://github.com/danielventurini/npm-diff} e a ferramenta \textit{compare}\footnote{https://github.com/danielventurini/cnlg/compare/1.1.0..1.1.1} do \textit{GitHub}. Com isso, foi possível verificar o que foi adicionado e removido do código do provedor -- até mesmo do cliente -- em um determinado intervalo de versões. Assim, conhecendo exatamente o que foi introduzido e removido em uma determinada \textit{release}, torna-se mais fácil categorizar o tipo de alteração.
\end{itemize}

\begin{figure}
    \centering
    \includegraphics[scale=0.45]{figuras/bc_documentation.jpeg}
    \caption{Documentação de uma \textit{breaking change} no \textit{CHANGELOG} e nas \textit{release-notes}}
    \label{fig:bc_documentation}
\end{figure}{}

Após descobrir as alterações que introduziram uma \textit{breaking change}, categorias foram criadas para agrupar as \textit{breaking changes}. Por exemplo, quando um erro tratava-se de uma alteração de \gls{API}, uma categoria chamada \textit{Função Renomeada} foi criada e as demais \textit{breaking changes} que possuem características comuns a essa também foram categorizadas como \textit{Função Renomeada}. Assim será possível quantificar cada uma das categorias e visualizar as mais comuns. E o mesmo processo foi realizado para as demais \textit{breaking changes}, sempre visando criar categorias da maneira mais genérica que agrupassem os erros semelhantes.

Então, para todas as \textit{releases} analisadas manualmente, foram salvas as seguintes informações para que fosse possível quantificar as \textit{breaking changes} e responder esta e as demais questões de pesquisas:

\begin{enumerate}
    \item Em que local o erro foi documentado: \textit{issue, changelog, pull-request} etc;
    \item Quem consertou o erro: cliente ou providor;
    \item Em qual nível do \textit{SEMVER} o erro foi reparado;
    \item Quanto tempo o erro levou até ser corrigido; e
    \item Por quantas \textit{releases} o erro persistiu.
\end{enumerate}{}

%---------------------------------------------------%
\section{RQ3. Como os pacotes clientes se recuperam das \textit{breaking changes}?}
\label{sec:rq3}

\subsection{Método}
\label{apr:rq3}
Uma vez que os clientes se recuperaram de um erro, há duas maneiras para se obter informações sobre esta recuperação. A primeira maneira é quando o provedor corrige seu código e o cliente apenas atualiza sua \textit{string} de versionamento no \textit{package.json}. Para o provedor consertar o erro, deve haver uma \textit{issue} no seu repositório. A segunda maneira é quando o próprio cliente conserta o código. Neste caso, o cliente pode corrigir o código do provedor e realizar um \textit{pull-request}. Também, o cliente pode alterar apenas o seu código para que execute normalmente com a \textit{release} do provedor que introduziu a \textit{breaking change}.

Todas as informações sobre esta questão de pesquisa foram recuperadas do \textit{GitHub}. As informações foram encontradas em \textit{CHANGELOGs, release-notes, issues} e \textit{pull-requests}. Os \textit{CHANGELOGs} contêm informações sobre os erros consertados. A partir das \textit{issues} é possível entender com os comentários dos clientes quais foram as ações que eles realizaram para se recuperar de uma determinada \textit{breaking change}. Pois, assim como o código de um pacote fica emaranhado com o código no restante do ecossistema ao qual ele pertence, o mesmo acontece com as \textit{issues}. Uma manifestação disso é que muitas \textit{issues} abertas em um projeto são vinculadas a \textit{issues} relacionadas, em projetos iguais ou diferentes, pois os desenvolvedores estão rastreando as causas de um problema \cite{Zhang:2018:WIL:3242887.3242891}. De maneira análoga, os \textit{pull-requests} que são relacionados ao mesmo problema também são marcados. Todas estas informações corroboram para descobrir como a \textit{breaking change} foi tratada/consertada e quem -- cliente ou provedor -- a consertou, caso tenha sido consertada.

Os \textit{commits} são alternativas para as \textit{issues} quando a busca se dá no repositório do cliente. Sobre os \textit{commits}, mensagens do tipo \textit{update dependencies, fix dependencies, fix errors} etc. sugerem que algum provedor foi atualizado para consertar algum erro ou um erro foi consertado diretamente no código do cliente. Estas informações são muito importantes, uma vez que o provedor corrigiu a \textit{breaking change} e o cliente apenas o atualizou. Assim, as mensagens dos \textit{commits} auxiliaram para descobrir os reais motivos da atualização -- ou retrocesso da versão.