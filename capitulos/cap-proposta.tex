% ATENÇÃO - veja com o seu orientador se você vai ter este capítulo e se este vai ter nome!
\chapter{Proposta}
\label{cap:proposta}

Esse capítulo é mais indicado para TCC 1, no qual o aluno pode expor melhor qual é a proposta de seus trabalho para a realização do TCC 1 e 2. Bem como o cronograma para realização das atividades.

%---------------------------------------------------%
\section{Cronograma de Atividades}
\label{cap:proposta:sec:cronograma}

(ATENÇÃO - Esta é apenas uma sugestão de elaboração de cronograma, veja com seu orientador!)

Em TCC 1 talvez seja interessante apresentar uma cronograma de realização das atividades da proposta que englobe as atividades do TCC 2.

Nesta seção são apresentadas as atividades a serem desenvolvidas para a execução da proposta. O cronograma de realização das tarefas é apresentado na Tabela~\ref{tab:cronograma}.

\begin{enumerate}
\item \textbf{Escrita do Projeto TCC 1.}
\item \textbf{Estudo de Técnicas...}
\item \textbf{Implementação da Ferramenta ...}
\item \textbf{Testes com o conjunto de \textit{benchmarks}.}
\item \textbf{Estudo de técnicas de Escalonamento de Tarefas.}
\item \textbf{Entrega do TCC 1}
\item \textbf{Apresentação do TCC 1}
\item \textbf{Realização de Experimentos.}
\item \textbf{Atividade do TCC 2}
\item \textbf{Escrita do TCC2}
\item \textbf{Entrega do TCC 2.}
\item \textbf{Apresentação do TCC 2.}
\end{enumerate}

\begin{table}[h!]
\renewcommand{\arraystretch}{1.3}
\caption{Cronograma de atividades}
\label{tab:cronograma}
\scalefont{0.9}
\begin{tabular}{|c|c|c|c|c|c|c|c|c|c|c|c|c|}
\hline
\multirow{2}{*}{\textbf{\textbf{Atividade}}} & \multicolumn{4}{c|}{\textbf{2014}}& \multicolumn{8}{c|}{\textbf{2015}} \\ \cline{2-13} 
& \multicolumn{1}{l|}{\textbf{Set}} & \multicolumn{1}{l|}{\textbf{Out}} & \multicolumn{1}{l|}{\textbf{Nov}} & \multicolumn{1}{l|}{\textbf{Dez}} & \multicolumn{1}{l|}{\textbf{Jan}} & \multicolumn{1}{l|}{\textbf{Fev}} & \multicolumn{1}{l|}{\textbf{Mar}} & \multicolumn{1}{l|}{\textbf{Abr}} & \multicolumn{1}{l|}{\textbf{Mai}} & \multicolumn{1}{l|}{\textbf{Jun}} & \multicolumn{1}{l|}{\textbf{Jul}} & \multicolumn{1}{l|}{\textbf{Ago}} \\ \hline
\textbf{1}  & X &   &   &   &   &   &   &   &   &   &   &  \\ \hline
\textbf{2}  & X & X & X & X &   &   &   &   &   &   &   &  \\ \hline
\textbf{3}  &   & X & X & X & X & X &   &   &   &   &   &  \\ \hline
\textbf{4}  &   &   & X & X & X & X &   & X & X &   &   &  \\ \hline
\textbf{5}  &   &   & X & X & X &   &   &   &   &   &   &  \\ \hline
\textbf{6}  &   &   & X & X & X & X & X & X & X & X &   &  \\ \hline
\textbf{7}  &   &   & X & X &   & X & X &   & X & X &   &  \\ \hline
\textbf{8}  &   &   &   & X & X &   & X & X &   & X & X &  \\ \hline
\textbf{9}  &   &   &   &   & X & X & X & X & X & X & X & X \\ \hline
\textbf{10} &   &   &   &   &   &   &   &   &   &   &   & X \\ \hline
\end{tabular}
\end{table}

%---------------------------------------------------%
\section{Considerações Finais}
\label{cap:proposta:consideracoes:finais}

Esta é uma sugestão de seção para dar um fechamento em cada uma dos capítulos.