\chapter{Conclusões}
\label{cap:conclusoes}

O reuso de código é um aspecto essencial no desenvolvimento de \textit{software}, principalmente no ecossistema do \textsf{npm}, onde os pacotes criam uma imensa rede de dependências. Entretanto as \textit{breaking changes} são uma consequência inevitável da facilidade que o reuso de código traz. \textit{Breaking changes} e seus impactos são estudados na literatura em vários ecossistemas de \textit{softwares} \cite{whyandhowluceneremovethiswords, intro:break_change, how_to_break_an_api, change_contracts}. Poucos estudos examinaram as \textit{breaking changes} no ecossistema do \textsf{npm} da perspectiva do pacote cliente, ou seja, executaram os casos de testes dos clientes para verificar o impacto das \textit{breaking changes} \cite{teorical_reference:bc_2, noregrets2018, using_others_tests}. Nesse trabalho foi analisado as \textit{breaking changes} no ecossistema do \textsf{npm} na perspectiva do pacote cliente e do pacote provedor.

Da perspectiva dos pacotes provedores foi analisado o quanto de \textit{breaking changes} são introduzidas que impactaram os seus pacotes clientes. Foi concluído que 11.7\% dos clientes são impactados por \textit{breaking changes}. Também, foram analisados os erros mais comuns realizados pelos provedores e que causam as \textit{breaking changes}. Foi verificado que o principal erro dos provedores é alterar seus comportamentos quando os clientes não esperam por essas alterações. Descobrimos que algumas alterações entre dois provedores podem os tornar incompatíveis e gerar uma \textit{breaking change}. Finalmente, concluímos que o número de casos de \textit{breaking changes} está aumentando ao longo dos anos e tende a continuar crescendo.

Da perspectiva dos pacotes cliente, analisados como eles se recuperaram das \textit{breaking changes}. Quando uma \textit{breaking change} é introduzida, os clientes se recuperam de várias maneiras. Observados que os pacotes clientes se recuperam das \textit{breaking changes} em 39.1\% dos casos realizando, principalmente, um \textit{upgrade} na versão dos provedores. Esse método é o mais rápido e simples para se recuperarem, mas os clientes demoram mais tempo para se recuperarem de uma \textit{breaking change} que os provedores. Ainda, as \textit{breaking changes} são introduzidas principalmente por provedores indiretos. Os pacotes clientes, ao se recuperarem das \textit{breaking changes}, preferem manter o \textit{range}, apenas atualizando a versão de seus provedores. Finalmente, quando as \textit{breaking changes} são documentadas, isso permite a recuperação dos clientes de maneira mais rápida.

Esse estudo contribui apresentando uma análise empírica sobre \textit{breaking changes} em \textit{releases minor} e \textit{patch} no ecossistema do \textsf{npm}. Foram realizadas várias análises aprofundadas nos repositórios dos clientes e dos provedores para garantir que todos os casos de \textit{breaking changes} são casos reais. As categorias criadas fornecem informações sobre os principais erros e os desenvolvedores podem usá-las como um guia para evitar tais erros em seus códigos. Finalmente, apresentamos várias sugestões sobre como os clientes e os provedores podem melhorar a qualidade de seus desenvolvimentos, prevenindo as \textit{breaking changes}.