\chapter{Referencial Teórico}
\label{cap:ref-teorico}

\section{\gls{NPM}}
\label{ref-teo:npm}
O \gls{NPM} é um projeto \textit{open-source} de gerenciamento de pacotes para o \textit{Node.js}. Lançado em 2009, seu principal objetivo é facilitar o compartilhamento de códigos escrito, principalmente, em \textit{Javascript}. Intitulada com o mesmo nome, a \gls{NPM} \textit{Inc} é um repositório online de publicação e compartilhamento de projetos. Atualmente, o \gls{NPM} ocupa a posição de maior repositório para uma dada linguagem com mais de 1 milhão de projetos\footnote{http://www.modulecounts.com/}, enquanto que o segundo maior repositório -- \textit{Maven} -- contém pouco mais de 200 mil  projetos.

O \gls{NPM} permite que, com apenas um simples comando, o usuário realize o download, publique, instale, desinstale pacotes diretamente do repositório do \gls{NPM} entre outras tarefas. A facilidade proporcionada pelo \gls{NPM} corrobora para a grande popularidade do \textit{Javascript} e para que o compartilhamento de biblioteca seja largamente utilizado, uma vez que 97\% dos aplicativos \textit{web} são oriundos do \textit{NPM}\footnote{https://blog.npmjs.org/post/180868064080/this-year-in-javascript-2018-in-review-and-npms}.
%In fact, 97\% of the code in a modern web application comes from npm

\section{Node.js}
\label{ref-teo:node}
O \textit{Node.js} é um projeto \textit{open-source} implementado em \textit{C++} sobre a \textit{engine Javascript V8} do \textit{Google}, que é um compilador \textit{Javascript} para \textit{web}. O \textit{Node.js} foi criado com o objetivo de estender o código \textit{Javascript} para além das páginas \textit{web}: agora o código \textit{Javascript} pode ser executado nos servidores. Com o \textit{Javascript} executando no \textit{front-end} e no \textit{back-end} não se faz necessário que os desenvolvedores saibam duas linguagens de programação diferentes, pois o \textit{Javascript} pode ser utilizado em ambos os contextos. A dualidade de utilização do \textit{Javascript} permitida pelo \textit{Node.js} foi um dos principais fatores que levou à grande popularidade do \textit{Javascript}, uma vez que o \textit{Node.js} é o \textit{framework} mais utilizado atualmente, de acordo com o \textit{Stack Overflow}\footnote{https://insights.stackoverflow.com/survey/2019\#technology-_-other-frameworks-libraries-and-tools}.

\section{\gls{SemVer}}
\label{ref-teo:semver}


\section{Working Tree}
\label{ref-teo:working_tree}

\section{Breaking Change}
\label{ref-teo:breaking_change}