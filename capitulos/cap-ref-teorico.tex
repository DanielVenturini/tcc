\chapter{Referencial Teórico}
\label{cap:ref-teorico}

\section{\gls{NPM}}
\label{ref-teo:npm}
O \gls{NPM} é um projeto \textit{open-source} de gerenciamento de pacotes para o \textit{Node.js}. Lançado em 2009, seu principal objetivo é facilitar o compartilhamento de códigos escrito, principalmente, em \textit{Javascript}. Intitulada com o mesmo nome, a \gls{NPM} \textit{Inc} é um repositório online de publicação e compartilhamento de projetos. Atualmente, o \gls{NPM} ocupa a posição de maior repositório para uma dada linguagem com mais de 1 milhão de projetos\footnote{http://www.modulecounts.com/}, enquanto que o segundo maior repositório -- \textit{Maven} -- contém pouco mais de 200 mil  projetos.

O \gls{NPM} permite que, com apenas um simples comando, o usuário realize o download, publique, instale, desinstale pacotes diretamente do repositório do \gls{NPM} entre outras tarefas.
%In fact, 97\% of the code in a modern web application comes from npm

\section{Node.js}
\label{ref-teo:node}


\section{\gls{SemVer}}
\label{ref-teo:semver}


\section{Working Tree}
\label{ref-teo:working_tree}

\section{Breaking Change}
\label{ref-teo:breaking_change}