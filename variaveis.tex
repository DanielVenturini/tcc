% Dados do curso. Caso seja BCC:
\program{Curso de Bacharelado em Ciência da Computação}
\programen{Undergradute Program in Computer Science}
\degree{Bacharel}
\degreearea{Ciência da Computação}
% Caso seja TSI:
% \program{Curso Superior de Tecnologia em Sistemas para Internet}
% \programen{Undergradute Program in Tecnology for Internet Systems}
% \degree{Tecnólogo}
% \degreearea{Tecnologia em Sistemas para Internet}


% Dados da disciplina. Escolha uma das opções e a descomente:
% TCC1:
\goal{Proposta de Trabalho de Conclusão de Curso de Graduação}
\course{Trabalho de Conclusão de Curso 2}
% TCC2:
% \goal{Trabalho de Conclusão de Curso de graduação}
% \course{Trabalho de Conclusão de Curso 2}


% Dados do TCC (precisa alterar)
\author{Daniel Venturini}  % Seu nome
\title{Estudo empírico sobre \textit{Breaking changes} no ecossistema do \textsf{npm}} % Título do trabalho
\titleen{An Empirical Study of Breaking Changes in the \textsf{npm} Ecosystem } % Título traduzido para inglês
\advisor{Prof. Dr. Ivanilton Polato} % Nome do orientador. Lembre-se de prefixar com "Prof. Dr.", "Profª. Drª.", "Prof. Me." ou "Profª. Me."}
\coadvisor{Prof. Dr. Igor Scaliante Wiese} % Nome do coorientador, caso exista. Caso não exista, comente a linha.
\depositshortdate{2020} % Ano em que depositou este documento

% Dados da ficha catalografica. Ela é opcional, mas é uma boa ideia inserí-la. Exemplos para geração (http://fichacatalografica.sibi.ufrj.br/)
\fichacatautor{}  % Nome conforme citado (ou seja, no formato "Sobrenome, Nome").
\fichacatbib{Biblioteca da UTFPR de Campo Mourão} % Não alterar
\fichacatpum{M488} % Código Cutter-Sanborn. Use a primeira letra do sobrenome seguido do número conforme as primeiras letras do sobrenome e a tabela http://www.amormino.com.br/cutter-sanborn/cutter1.html
\fichacatpalcha{} % Assuntos do trabalho. Cada item deve ser enumerado e separado por ponto: 1. xxx. 2. yyy. 3. zzz.
\fichacatpdois{} % Deixar em branco