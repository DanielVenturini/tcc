\documentclass[12pt,english,brazil,a4paper,utf8,oneside]{utfpr-tcc}

%Reviewing
\newcommand{\filipe}[1]{\textcolor{blue}{{\it [Filipe: #1]}}}
% copying from you, Filipe, hahaa
\newcommand{\daniel}[1]{\textcolor{red}{{\it [Daniel: #1]}}}

% carrega o arquivo configuracoes.tex que contém os pacotes e comandos Latex.
%
% Esse arquivo conterá pacotes e comandos utilizados na monografia
%
% Observação - devido a um erro do sharelatex foi necessário colocar na raiz do projeto os seguintes arquivos:
% gcnumparser.sty, fcprefix.sty, fmtcount.sty, fc-poruges.def, fcportuguese.def.
% Tal problema foi relatado em: https://github.com/nlct/fmtcount/issues/26
% Quando o sharelatex corrigir o problema acredito que podemos remover esses arquivos do projeto. At. Luiz Arthur.
%

% Este comando não é necessário: utilizei apenas para deixar o latex2rtf
% feliz (e descobrir a codificação do texto).
\usepackage[utf8]{inputenc}

% Suporte a figuras e subfiguras
\usepackage{graphics}
\usepackage{subfigure}

% Suporte a tabelas (principalmente do cronograma)
\usepackage{tabularx}
\usepackage{multirow}
\usepackage{array}
\usepackage{tabularx}
\usepackage{colortbl}
\usepackage{hhline}
\usepackage{xcolor}


% better tables
\usepackage{booktabs}

% frame box
\usepackage{mdframed}

% Escalar fontes para redimencionar, por exemplo tabelas
\usepackage{scalefnt}

% Algoritmos.
\usepackage{algorithm,algorithmic}

\usepackage[alf]{abntex2cite}

% the table's styles
\usepackage{booktabs}

% Elementos geralmente utilizados na tabela do cronograma
\newcommand{\fullcell}{\multicolumn{1}{>{\columncolor[gray]{0.5}}c}{}}
\newcommand{\fullcellline}{\multicolumn{1}{>{\columncolor[gray]{0.5}}c|}{}}
\newcommand{\mc}[3]{\multicolumn{#1}{#2}{#3}}
\newcommand{\y}{\rule{8pt}{4pt}}
\newcommand{\n}{\hspace*{8pt}}

% Define o caminho das figuras
\graphicspath{{images/}}

%% Configuração de glossário
\usepackage[portuguese]{nomencl}
\usepackage[nogroupskip,acronym,nomain,nonumberlist,nopostdot,nohypertypes={acronym}]{glossaries}

\makenoidxglossaries

% para siglas em português
\newcommand{\sigla}[2]
{
 \newglossaryentry{#1}{
  name=#1,
  description={#2},
  first={#2 (#1)},
  long={#2}
 }  
}

% para siglas de língua estrangeira, nessas a descrição longa fica em itálico.
\newcommand{\siglaIt}[2]
{
 \newglossaryentry{#1}{
  name=#1,
  description={\textit{#2}},
  first={\textit{#2} ({#1})},
  long={\textit{#2}}
 }  
}

% --- Estilos para apresentação de Código ----- %
\usepackage{listings}
\lstset{escapechar=§}
\lstloadaspects{formats}

\lstset{
	aboveskip=0cm,
	stringstyle=\ttfamily,
	showstringspaces = false,
	basicstyle=\scriptsize\ttfamily,
	commentstyle=\color{gray!45},
	keywordstyle=\bfseries,
	ndkeywordstyle=\bfseries,
	identifierstyle=\ttfamily,
	numbers=left,
	numbersep=15pt,
	numberstyle=\tiny,
	numberfirstline = false,
	breaklines=true
}

\lstdefinelanguage{JavaScript}{
	keywords={typeof, new, true, false, catch, function, return, null, catch, switch, var, const, let, async, await, if, in, while, do, else, case, break, from},
	ndkeywords={class, export, boolean, throw, implements, import, this},
	sensitive=false,
	comment=[l]{//},
	morecomment=[s]{/*}{*/},
	morestring=[b]',
	morestring=[b]"
}

% Diff language
\usepackage{xcolor}
\definecolor{diffstart}{named}{lightgray}
\definecolor{diffincl}{named}{blue}
\definecolor{diffrem}{named}{red}

\lstset{
	aboveskip=0cm,
	stringstyle=\scriptsize,
	showstringspaces = false,
	basicstyle=\scriptsize\ttfamily,
	commentstyle=\color{gray!45},
	keywordstyle=\bfseries,
	ndkeywordstyle=\bfseries,
	identifierstyle=\ttfamily,
	numbers=left,
	numbersep=15pt,
	numberstyle=\tiny,
	numberfirstline = false,
	breaklines=true
}

\lstdefinelanguage{diff}{
    basicstyle=\scriptsize\ttfamily,
	morecomment=[f][\color{diffstart}]{@@},
	morecomment=[f][\color{diffincl}]{+\ },
	morecomment=[f][\color{diffrem}]{-\ },
	morestring=[b]',
	morestring=[b]",
	keywords={typeof, new, true, false, catch, function, return, null, catch, switch, var, const, let, async, await, if, in, while, do, else, case, break, from},		
	ndkeywords={class, export, boolean, throw, implements, import, this},
}
% --- Fim da Definição de Estilos para apresentação de Código ----- %
% carrega o arquivo constantes.tex que contém dados do curso/monografia que NÃO DEVEM ser alterados. 
% Dados do curso que não precisam de alteração
\university{Universidade Tecnológica Federal do Paraná}
\universityen{Federal University of Technology -- Paraná}
\universityunit{Departamento Acadêmico de Computação}
\address{Campo Mourão}
\addressen{Campo Mourão, PR, Brazil}
\documenttype{Monografia}
\documenttypeen{Monograph}
\degreetype{Graduação}
% carrega o arquivo variaveis.tex que contém dados do acadêmico/monografia que DEVEM ser alterados.
% Dados do curso. Caso seja BCC:
\program{Curso de Bacharelado em Ciência da Computação}
\programen{Undergradute Program in Computer Science}
\degree{Bacharel}
\degreearea{Ciência da Computação}
% Caso seja TSI:
% \program{Curso Superior de Tecnologia em Sistemas para Internet}
% \programen{Undergradute Program in Tecnology for Internet Systems}
% \degree{Tecnólogo}
% \degreearea{Tecnologia em Sistemas para Internet}


% Dados da disciplina. Escolha uma das opções e a descomente:
% TCC1:
\goal{Proposta de Trabalho de Conclusão de Curso de Graduação}
\course{Trabalho de Conclusão de Curso 1}
% TCC2:
% \goal{Trabalho de Conclusão de Curso de graduação}
% \course{Trabalho de Conclusão de Curso 2}


% Dados do TCC (precisa alterar)
\author{Daniel Venturini}  % Seu nome
\title{Estudo empírico sobre \textit{Breaking changes} no ecossistema do \textit{npm}} % Título do trabalho
\titleen{An Empirical Study of Breaking Changes in the npm Ecosystem } % Título traduzido para inglês
\advisor{Prof. Dr. Ivanilton Polato} % Nome do orientador. Lembre-se de prefixar com "Prof. Dr.", "Profª. Drª.", "Prof. Me." ou "Profª. Me."}
\coadvisor{Prof. Dr. Igor Scaliante Wiese} % Nome do coorientador, caso exista. Caso não exista, comente a linha.
\depositshortdate{2019} % Ano em que depositou este documento

% Dados da ficha catalografica. Ela é opcional, mas é uma boa ideia inserí-la. Exemplos para geração (http://fichacatalografica.sibi.ufrj.br/)
\fichacatautor{}  % Nome conforme citado (ou seja, no formato "Sobrenome, Nome").
\fichacatbib{Biblioteca da UTFPR de Campo Mourão} % Não alterar
\fichacatpum{M488} % Código Cutter-Sanborn. Use a primeira letra do sobrenome seguido do número conforme as primeiras letras do sobrenome e a tabela http://www.amormino.com.br/cutter-sanborn/cutter1.html
\fichacatpalcha{} % Assuntos do trabalho. Cada item deve ser enumerado e separado por ponto: 1. xxx. 2. yyy. 3. zzz.
\fichacatpdois{} % Deixar em branco

% carrega o arquivo listaabreviaturas.tex que está dentro do diretório pretextual, esse arquivo contém as siglas utilizadas na monografia.
% quando a sigla for de língua portuguesa utilize \sigla{SIGLA}{Significado em português}
% quando a sigla for de língua estrangeira utilize \siglaIt{SIGLA}{Significado em Inglês}

\sigla{UTFPR}{Universidade Tecnológica Federal do Paraná}
\siglaIt{ACM}{Association for Computing Machinery}
\siglaIt{IP}{Internet Protocol}
\siglaIt{TCP}{Transmission Control Protocol}


% No texto quando for utilizar a sigla utilize os seguintes comandos:
%\acrlong{label} - acronimo/sigla longo
%\acrshort{label} - acronimo/sigla curta
%\Gls{TCP} - sigla com o significado primeiro em Maiusculo
%\GLS{TCP} - sigla com o significado tudo em MAIUSCULO
%\gls{TCP} - sigla com o significado tudo em minusculo % usando glossaries

\begin{document}

\frontmatter
\maketitle

% Dedicatória é opcional, para usar descomentar a linha a seguir e edite o arquivo pretextual/dedicatoria.tex
%\dedicate{Para minha mãe, para meu pai e para você...} % Opcional - descomentar para usar

% Agradecimento é opcional, para usar descomentar a linha a seguir e edite o arquivo pretextual/dedicatoria.tex
%\begin{agradecimentos}

agradeço agradeço agradeço agradeço agradeço agradeço agradeço agradeço  agradeço agradeço agradeço agradeço agradeço agradeço agradeço agradeço agradeço agradeço agradeço agradeço agradeço agradeço agradeço agradeço agradeço agradeço agradeço agradeço agradeço agradeço agradeço agradeço agradeço agradeço agradeço agradeço agradeço agradeço agradeço agradeço agradeço agradeço agradeço agradeço agradeço agradeço agradeço agradeço agradeço agradeço agradeço agradeço agradeço agradeço agradeço agradeço agradeço agradeço agradeço agradeço agradeço agradeço agradeço agradeço agradeço agradeço agradeço agradeço agradeço agradeço agradeço agradeço 

\end{agradecimentos} % Opcional - descomentar para usar

% carrega o arquivo resumo.tex que está dentro do diretório pretextual, esse arquivo deve conter o resumo da monografia.
\begin{resumo}
%Elemento obrigatório, constituído de uma sequência de frases concisas e objetivas, em forma de texto.  Deve apresentar os objetivos, métodos empregados, resultados e conclusões.  O resumo deve ser redigido em parágrafo único, conter no máximo 500 palavras e ser seguido dos termos representativos do conteúdo do trabalho (palavras-chave).

% TODO: se possível, escreva um resumo estruturado. Para TCC 1, o resumo estruturado teria os seguintes elementos:
\textbf{Contexto:} o \textit{npm} é largamente utilizado e é o maior repositório para uma dada linguagem. Os pacotes hospedados no \textit{npm} dependem um dos outros, criando uma rede de interconectividade entre eles. Entretanto, os provedores evoluem independentemente dos seus clientes e, por vezes, introduzem alterações que podem causar um comportamento inesperado nos clientes. Essas alterações são as \textit{breaking changes} e se tornam um problema quando os clientes as recebem, mas não deveriam receber.\\
\textbf{Objetivo:} este trabalho propõe mensurar e categorizar as \textit{breaking changes} e analisar como os clientes se recuperam delas.\\
\textbf{Método:} de uma amostra dos pacotes do \textit{npm}, copiá-los localmente, resolver a versão dos seus provedores para a última versão disponível no momento da \textit{release} do cliente. Posteriormente, executar o pacote através dos \textit{scripts npm install/npm test}. Então, para cada \textit{release} do cliente que resultou em erro, verificar no código da \textit{release} e no repositório do provedor para confirmar se o erro foi causado pelo provedor, sendo então uma \textit{breaking change}.\\
% \textbf{Resultados esperados:} 
% ou, para TCC 2:
% \textbf{Contexto:} \\
% \textbf{Objetivo:} \\
% \textbf{Método:} \\
% \textbf{Resultados:} \\
% \textbf{Conclusões:}

% Palavras-chaves, separadas por ponto (tente não definir mais do que cinco)
\palavraschaves{\textit{npm}. \textit{Breaking change}. Versionamento Semântico. Dependências}
\end{resumo}
% carrega o arquivo abstract.tex que está dentro do diretório pretextual, esse arquivo deve conter um resumo escrito na linguá inglesa para a monografia.
%% Caso seja TCC 2, precisa traduzir o resumo e as palavras-chaves para inglês:
\begin{abstract}

\textbf{Context:} Packages hosted on \textsf{npm} create a dependency network, where \textit{client} packages are the ones that depends on \textit{provider} packages. Occasionally, providers introduce breaking changes, which are changes that may cause defects on clients. These changes should be only introduced in \textit{major} level of Semantic Versioning, but when introduced in \textit{minor} or \textit{patch} levels, these may cause issues on clients. \textbf{Objective:} This work proposes a study about breaking changes in minor and patch levels on \textsf{npm}. Our objectives are: 1) to measure and 2) to categorize the breaking changes, and 3) to analyze how clients recover themselves. \textbf{Method:} From a sample of clients from \textsf{npm}, we restored the releases and installed the latest version of providers that the client accepted in the release timestamp. Following, we executed the \texttt{npm install/test} scripts. All releases that raised an error were analyzed, and the client and providers code and repositories was verified to check whether the error was raised by a provider, characterizing a breaking change. \textbf{Results:} Altogether, 55 providers introduced breaking changes that impacted 13.9\% of client releases and these breaking changes have increased 63.4\% from the respective previous year. Yet, 54.9\% of provider releases with breaking changes have more commits than their other releases. Breaking changes are introduced in minor and patch level in the same proportion, but the majority is fixed by providers in patch levels and are documented in 78.1\% of cases, mainly on issues, causing the fix to be 3.3 time faster. While indirect providers are the ones that introduces the majority of breaking changes, clients fix these in 39.1\% of cases and they prefer to do an upgrade on the provider's version without changing the range. \textbf{Conclusions:} Breaking changes do really happen in minor and patch releases. Previous studies focused only on API breaking changes, while this study used clients' tests to find any types of breaking changes. We presented several suggestions to developers to improve their interaction with the \textsf{npm} ecosystem.

% Palavras-chaves em inglês, separadas por ponto.
\keywords{\textsf{npm}. Breaking change. Semantic Version. Dependency management.}
\end{abstract}

% Listas (opcionais, mas recomenda-se a partir de 5 elementos)
\listoffigures
\listoftables
%\listofacronyms
\printnoidxglossaries

% Sumário
\tableofcontents

\mainmatter

% Capítulos da monografia:
\chapter{Introdução}
\label{cap:introducao}

Coloque aqui o texto da introdução, contextualizando o seu trabalho...

Testando o uso das siglas na - pela primeira vez para \gls{ACM}. Segunda vez para \gls{ACM}...

blabla... \gls{UTFPR}

xxx \gls{TCP}

\acrlong{UTFPR}

A rede \gls{IP}...

% Sugestões de seções
\section{Considerações preliminares}

Aqui você pode descrever problemas, soluções e outros assuntos que ajudem a introduzir o leitor ao contexto de seu trabalho...

(ATENÇÃO - Essa seção é uma sugestão, veja com o seu orientador se você vai ter essa e se vai ter esse nome!)

TEXTO TEXTO TEXTO TEXTO TEXTO TEXTO TEXTO TEXTO TEXTO TEXTO TEXTO TEXTO TEXTO TEXTO TEXTO TEXTO TEXTO TEXTO TEXTO TEXTO TEXTO TEXTO TEXTO TEXTO TEXTO TEXTO TEXTO TEXTO TEXTO TEXTO TEXTO TEXTO TEXTO TEXTO TEXTO TEXTO TEXTO TEXTO TEXTO TEXTO TEXTO TEXTO TEXTO TEXTO TEXTO TEXTO TEXTO TEXTO TEXTO TEXTO TEXTO TEXTO TEXTO TEXTO TEXTO TEXTO TEXTO TEXTO TEXTO TEXTO TEXTO TEXTO TEXTO TEXTO TEXTO TEXTO TEXTO TEXTO TEXTO TEXTO TEXTO TEXTO TEXTO TEXTO TEXTO TEXTO TEXTO TEXTO TEXTO TEXTO TEXTO TEXTO TEXTO TEXTO TEXTO TEXTO TEXTO TEXTO TEXTO TEXTO TEXTO TEXTO TEXTO TEXTO TEXTO TEXTO TEXTO TEXTO TEXTO TEXTO TEXTO TEXTO TEXTO TEXTO TEXTO TEXTO TEXTO TEXTO TEXTO TEXTO TEXTO TEXTO TEXTO TEXTO TEXTO TEXTO TEXTO TEXTO TEXTO TEXTO TEXTO TEXTO TEXTO TEXTO TEXTO TEXTO TEXTO TEXTO TEXTO TEXTO TEXTO TEXTO

\section{Problema de Pesquisa}
\label{cap:introducao:sec:problema:pesquisa}

Descreva o seu Problema de Pesquisa com as questões de pesquisa que seu trabalho irá tentar responder.

\section{Objetivos}
\label{cap:introducao:sec:objetivos}

Descreva de maneira sucinta os objetivos de seu trabalho (o que você fará durante o desenvolvimento de TCC 1 e TCC 2?). Faça um texto BEM curto e objetivo...

(ATENÇÃO - Essa seção é uma sugestão, veja com o seu orientador se você vai ter essa e se vai ter esse nome!)

TEXTO TEXTO TEXTO TEXTO TEXTO TEXTO TEXTO TEXTO TEXTO TEXTO TEXTO TEXTO TEXTO TEXTO TEXTO TEXTO TEXTO TEXTO TEXTO TEXTO TEXTO TEXTO TEXTO TEXTO TEXTO TEXTO TEXTO TEXTO TEXTO TEXTO TEXTO TEXTO TEXTO TEXTO TEXTO TEXTO TEXTO TEXTO TEXTO TEXTO TEXTO TEXTO TEXTO TEXTO TEXTO TEXTO TEXTO TEXTO TEXTO TEXTO TEXTO TEXTO TEXTO TEXTO TEXTO TEXTO TEXTO TEXTO TEXTO TEXTO TEXTO TEXTO TEXTO TEXTO TEXTO TEXTO TEXTO TEXTO TEXTO TEXTO TEXTO TEXTO TEXTO TEXTO TEXTO TEXTO TEXTO TEXTO TEXTO TEXTO TEXTO TEXTO TEXTO TEXTO TEXTO TEXTO TEXTO TEXTO TEXTO TEXTO TEXTO TEXTO TEXTO TEXTO TEXTO TEXTO TEXTO TEXTO TEXTO TEXTO TEXTO TEXTO TEXTO TEXTO TEXTO TEXTO TEXTO TEXTO TEXTO TEXTO TEXTO TEXTO TEXTO TEXTO TEXTO TEXTO TEXTO TEXTO TEXTO TEXTO TEXTO TEXTO TEXTO TEXTO TEXTO TEXTO TEXTO TEXTO TEXTO TEXTO TEXTO TEXTO

\section{Contribuições}
\label{cap:introducao:sec:contribuicoes}

No que o seu trabalho ajuda? Há diferenças entre o seu trabalho e outros?

(ATENÇÃO - Essa seção é uma sugestão, veja com o seu orientador se você vai ter essa e se vai ter esse nome!)

TEXTO TEXTO TEXTO TEXTO TEXTO TEXTO TEXTO TEXTO TEXTO TEXTO TEXTO TEXTO TEXTO TEXTO TEXTO TEXTO TEXTO TEXTO TEXTO TEXTO TEXTO TEXTO TEXTO TEXTO TEXTO TEXTO TEXTO TEXTO TEXTO TEXTO TEXTO TEXTO TEXTO TEXTO TEXTO TEXTO TEXTO TEXTO TEXTO TEXTO TEXTO TEXTO TEXTO TEXTO TEXTO TEXTO TEXTO TEXTO TEXTO TEXTO TEXTO TEXTO TEXTO TEXTO TEXTO TEXTO TEXTO TEXTO TEXTO TEXTO TEXTO TEXTO TEXTO TEXTO TEXTO TEXTO TEXTO TEXTO TEXTO TEXTO TEXTO TEXTO TEXTO TEXTO TEXTO TEXTO TEXTO TEXTO TEXTO TEXTO TEXTO TEXTO TEXTO TEXTO TEXTO TEXTO TEXTO TEXTO TEXTO TEXTO TEXTO TEXTO TEXTO TEXTO TEXTO TEXTO TEXTO TEXTO TEXTO TEXTO TEXTO TEXTO TEXTO TEXTO TEXTO TEXTO TEXTO TEXTO TEXTO TEXTO TEXTO TEXTO TEXTO TEXTO TEXTO TEXTO TEXTO TEXTO TEXTO TEXTO TEXTO TEXTO TEXTO TEXTO TEXTO TEXTO TEXTO TEXTO TEXTO TEXTO TEXTO TEXTO

\section{Organização do Texto}
\label{cap:introducao:sec:organizacao:texto}

No Capítulo~\ref{cap:introducao} blablabla, no capítulo seguinte tititi, etc... Nossa proposta é apresentada no Capítulo~\ref{cap:proposta}.... Finalmente, no Capítulo~\ref{cap:conclusoes} apresentamos as conclusões obtidas no desenvolvimento deste trabalho...

(ATENÇÃO - Essa seção é uma sugestão, veja com o seu orientador se você vai ter essa e se vai ter esse nome!)

TEXTO TEXTO TEXTO TEXTO TEXTO TEXTO TEXTO TEXTO TEXTO TEXTO TEXTO TEXTO TEXTO TEXTO TEXTO TEXTO TEXTO TEXTO TEXTO TEXTO TEXTO TEXTO TEXTO TEXTO TEXTO TEXTO TEXTO TEXTO TEXTO TEXTO TEXTO TEXTO TEXTO TEXTO TEXTO TEXTO TEXTO TEXTO TEXTO TEXTO TEXTO TEXTO TEXTO TEXTO TEXTO TEXTO TEXTO TEXTO TEXTO TEXTO TEXTO TEXTO TEXTO TEXTO TEXTO TEXTO TEXTO TEXTO TEXTO TEXTO TEXTO TEXTO TEXTO TEXTO TEXTO TEXTO TEXTO TEXTO TEXTO TEXTO TEXTO TEXTO TEXTO TEXTO TEXTO TEXTO TEXTO TEXTO TEXTO TEXTO TEXTO TEXTO TEXTO TEXTO TEXTO TEXTO TEXTO TEXTO TEXTO TEXTO TEXTO TEXTO TEXTO TEXTO TEXTO TEXTO TEXTO TEXTO TEXTO TEXTO TEXTO TEXTO TEXTO TEXTO TEXTO TEXTO TEXTO TEXTO TEXTO TEXTO TEXTO TEXTO TEXTO TEXTO TEXTO TEXTO TEXTO TEXTO TEXTO TEXTO TEXTO TEXTO TEXTO TEXTO TEXTO TEXTO TEXTO TEXTO TEXTO TEXTO TEXTO TEXTO
\chapter{Referencial Teórico}
\label{cap:ref-teorico}
 % Esse capítulo e nome é apenas uma sugestão.
%% ATENÇÃO - veja com o seu orientador se você vai ter este capítulo e se este vai ter nome!
\chapter{Trabalhos Relacionados}
\label{cap:trabalhos:relacionados}

Apresente aqui os trabalhos similares ao seu trabalho ou que são importantes para o entendimento do seu trabalho...

\section{Uso de citações}
\label{cap:trabalhos:sec:relacionados:uso:citacoes}

Este é um exemplo do uso de citações no texto \cite{tomasulo:algorithm:5392028}.

Segundo \citeonline[p.~56]{Moore:2000:CMC:333067.333074} para citações textuais...

De acordo com o trabalho de \citeonline{Moore:2000:CMC:333067.333074} para citações textuais não tão específicas...

%---------------------------------------------------%
\section{Considerações Finais}
\label{cap:trabalhos:relacionados:sec:consideracoes:finais}

Esta é uma sugestão de seção para dar um fechamento em cada uma dos capítulos. % Esse capítulo e nome é apenas uma sugestão.
%% ATENÇÃO - veja com o seu orientador se você vai ter este capítulo e se este vai ter nome!
\chapter{Proposta}
\label{cap:proposta}

Esse capítulo é mais indicado para TCC 1, no qual o aluno pode expor melhor qual é a proposta de seus trabalho para a realização do TCC 1 e 2. Bem como o cronograma para realização das atividades.

%---------------------------------------------------%
\section{Cronograma de Atividades}
\label{cap:proposta:sec:cronograma}

(ATENÇÃO - Esta é apenas uma sugestão de elaboração de cronograma, veja com seu orientador!)

Em TCC 1 talvez seja interessante apresentar uma cronograma de realização das atividades da proposta que englobe as atividades do TCC 2.

Nesta seção são apresentadas as atividades a serem desenvolvidas para a execução da proposta. O cronograma de realização das tarefas é apresentado na Tabela~\ref{tab:cronograma}.

\begin{enumerate}
\item \textbf{Escrita do Projeto TCC 1.}
\item \textbf{Estudo de Técnicas...}
\item \textbf{Implementação da Ferramenta ...}
\item \textbf{Testes com o conjunto de \textit{benchmarks}.}
\item \textbf{Estudo de técnicas de Escalonamento de Tarefas.}
\item \textbf{Entrega do TCC 1}
\item \textbf{Apresentação do TCC 1}
\item \textbf{Realização de Experimentos.}
\item \textbf{Atividade do TCC 2}
\item \textbf{Escrita do TCC2}
\item \textbf{Entrega do TCC 2.}
\item \textbf{Apresentação do TCC 2.}
\end{enumerate}

\begin{table}[h!]
\renewcommand{\arraystretch}{1.3}
\caption{Cronograma de atividades}
\label{tab:cronograma}
\scalefont{0.9}
\begin{tabular}{|c|c|c|c|c|c|c|c|c|c|c|c|c|}
\hline
\multirow{2}{*}{\textbf{\textbf{Atividade}}} & \multicolumn{4}{c|}{\textbf{2014}}& \multicolumn{8}{c|}{\textbf{2015}} \\ \cline{2-13} 
& \multicolumn{1}{l|}{\textbf{Set}} & \multicolumn{1}{l|}{\textbf{Out}} & \multicolumn{1}{l|}{\textbf{Nov}} & \multicolumn{1}{l|}{\textbf{Dez}} & \multicolumn{1}{l|}{\textbf{Jan}} & \multicolumn{1}{l|}{\textbf{Fev}} & \multicolumn{1}{l|}{\textbf{Mar}} & \multicolumn{1}{l|}{\textbf{Abr}} & \multicolumn{1}{l|}{\textbf{Mai}} & \multicolumn{1}{l|}{\textbf{Jun}} & \multicolumn{1}{l|}{\textbf{Jul}} & \multicolumn{1}{l|}{\textbf{Ago}} \\ \hline
\textbf{1}  & X &   &   &   &   &   &   &   &   &   &   &  \\ \hline
\textbf{2}  & X & X & X & X &   &   &   &   &   &   &   &  \\ \hline
\textbf{3}  &   & X & X & X & X & X &   &   &   &   &   &  \\ \hline
\textbf{4}  &   &   & X & X & X & X &   & X & X &   &   &  \\ \hline
\textbf{5}  &   &   & X & X & X &   &   &   &   &   &   &  \\ \hline
\textbf{6}  &   &   & X & X & X & X & X & X & X & X &   &  \\ \hline
\textbf{7}  &   &   & X & X &   & X & X &   & X & X &   &  \\ \hline
\textbf{8}  &   &   &   & X & X &   & X & X &   & X & X &  \\ \hline
\textbf{9}  &   &   &   &   & X & X & X & X & X & X & X & X \\ \hline
\textbf{10} &   &   &   &   &   &   &   &   &   &   &   & X \\ \hline
\end{tabular}
\end{table}

%---------------------------------------------------%
\section{Considerações Finais}
\label{cap:proposta:consideracoes:finais}

Esta é uma sugestão de seção para dar um fechamento em cada uma dos capítulos. % Esse capítulo e nome é apenas uma sugestão (bom para TCC 1).
\chapter{Questões de Pesquisa}
\label{cap:qp}

Este trabalho propõe um estudo sobre as \textit{breaking changes} em \textit{releases minor} e \textit{patch} e seus impactos no ecossistema do \textsf{npm}. Para isso, três questões de pesquisa foram desenvolvidas para que seja possível executar o estudo. A seguir, há a motivação para cada questão de pesquisa. Nesta Seção estão descritos os métodos utilizados para responder cada uma das questões de pesquisa.

%---------------------------------------------------%
%----------------------RQ1--------------------------%
%---------------------------------------------------%

\section{QP1. Com que frequência \textit{breaking changes} impactam os clientes?}
\label{sec:qp1}

\subsubsection{Motivação}
\label{sec:qp1:motivation}
No ecossistema do \textsf{npm}, uma \textit{release} que contenha um erro pode afetar uma grande quantidade de pacotes, uma vez que a rede de dependências do \textsf{npm} é relativamente densa \cite{teorical_reference:npm_2}. Para evitar que \textit{breaking changes} se manifestem nos clientes, os provedores introduzem as \textit{breaking changes} em \textit{releases major}, seguindo o padrão do Versionamento Semântico, e os clientes podem utilizar \textit{strings semver} para aceitar apenas as versões \textit{minor} e \textit{patch} dos provedores -- o que é o padrão do \textsf{npm}. Entretanto, nem sempre o provedor é capaz de distinguir se suas alterações são ou não \textit{breaking changes} \cite{noregrets2018}, ou, muitas vezes, as \textit{breaking changes} são introduzidas sem que os provedores percebam. Estudos anteriores têm estudado \textit{breaking changes} no ecossistema do \textsf{npm} \cite{using_others_tests, noregrets2018, intro:break_change, teorical_reference:bc_1}, mas não estudaram a frequência e como se manifestam. Nesta RQ, serão quantificadas as manifestações das \textit{breaking changes} nos clientes.

\subsubsection{Método}
\label{sec:qp1:approach}

Quando os comandos \texttt{npm install} e \texttt{npm test} resultaram em erro, o nosso objetivo tornou-se em distinguir se o erro foi causado pelo provedor, caracterizando uma \textit{breaking change}, ou se foi causado apenas pelo próprio cliente, não sendo uma \textit{breaking change}. A primeira evidência é o \textit{stack trace} gerado pelo \textsf{npm} quando ocorre um erro. O \textit{stack trace} contém o tipo do erro, a localização exata do erro e o fluxo de execução no momento em que o erro ocorreu. Uma das principais informações são os provedores que estavam sendo executados no fluxo de execução. Se houve algum provedor no \textit{stack trace}, provavelmente o erro se tratava de uma \textit{breaking change}. Quando não houve algum provedor no \textit{stack trace}, provavelmente o erro não se tratava de uma \textit{breaking change}, mas ainda sim foram feitos os métodos descritos abaixo para confirmar se um erro era de fato uma \textit{breaking change} ou não.

Para quantificar as \textit{breaking changes}, foi necessário diferenciar entre um erro causado pelo próprio cliente, no qual não houve influência de nenhum provedor, e um erro causado por algum dos provedores, sendo assim uma \textit{breaking change}. Para realizar esta diferenciação, foram realizadas as seguintes heurísticas:

\begin{itemize}
    \item \textbf{Alterações nos códigos}: foi realizado algumas alterações nos códigos do cliente e do provedor para analisar o fluxo de execução até gerar o erro. Por exemplo, foi adicionado chamadas para \texttt{console.trace()} para visualizar a pilha de execução até essa chamada. Também, a chamada para \texttt{console.log()} foi muito utilizada para verificar o conteúdo das variáveis em tempo de execução e suas tipagens. Isso tudo para verificar como as variáveis estavam se comportando e como estavam sendo alteradas pelos provedores e pelo próprio cliente.

    \item \textbf{Sistemas integrados ao \textsf{GitHub}:} sistemas integrados\footnote{https://strongloop.com/strongblog/node-js-travis-circle-
codeship-compare/} são sistemas que integram-se aos repositórios e provêm tarefas automáticas para os desenvolvedores, tal como execução de testes. Esses sistemas integrados desempenharam um papel fundamental na análise manual. Se o \textit{status} do teste do \textit{commit} da \textit{release} do cliente nesses sistemas integrados estavam como sucesso e em nosso estudo foi identificado como um erro, provavelmente o erro se tratava de uma \textit{breaking change}. Isso pois o \textit{commit} da \textit{release} do cliente no sistema integrado era o mesmo \textit{commit} executado em nosso estudo. Assim, apenas a versão dos provedores poderia ter sido alterada e causado o erro.

    \item \textbf{\textit{Commits} do cliente:} foi analisado manualmente os \textit{commits} do cliente a partir do \textit{commit} da \textit{release} para verificar se o cliente tentou consertar algo em seu código. Se sim, foi realizado as alterações feitas pelo cliente para verificar se o erro foi consertado e concluir se o erro foi causado apenas pelo pacote cliente ou se foi causado por um dos pacotes provedores. Por exemplo, se um cliente atualizou um provedor e foi impactado por uma \textit{breaking change}, nos próximos \textit{commits} o cliente poderia realizar um \textit{downgrade} na versão do provedor ou realizar uma alteração em seu código para se recuperar da \textit{breaking change}. Os \textit{commits} nomeadas como \textit{"downgrade provider"}, \textit{"fix break change"}, \textit{"Bump tests and dependencies"} indicam que o cliente realizou alguma alteração para, provavelmente, se recuperar da \textit{breaking change}.

    \item \textbf{\textit{Issues/Pull-requests}:} se o erro é uma \textit{breaking change}, outros clientes podem ter sido impactados e provavelmente já foi documentada em uma \textit{issue} ou um \textit{pull-request}. Através dos comentários das \textit{issues}/\textit{pull-requests} foi possível recuperar informações detalhadas sobre o erro, qual provedor introduziu, se foi consertada etc. \textit{Issues} e \textit{pull-requests} foram muito importantes e permitiram encontrar muitas informações porque muitas \textit{issues} e \textit{pull-requests} referenciam outras, no mesmo projeto ou em projetos distintos, enquanto os desenvolvedores estão rastreando um erro \cite{Zhang:2018:WIL:3242887.3242891}.

    \item \textbf{\textit{Releases} precedentes e posteriores do provedor:} essa foi uma etapa muito importante para detectar se um erro era uma \textit{breaking change}. Se um erro é uma \textit{breaking change}, as \textit{releases} precedentes e posteriores do provedor poderiam consertar o erro. Nesse caso, foi desinstalado a \textit{release} atual e instalado uma \textit{release} anterior ou posterior daquela que causou o erro. Por fim, o \textit{script} de teste foi reexecutado. Por exemplo, se um cliente especificou um provedor \texttt{p} como \texttt{\{"p": "\textasciicircum1.0.2"\}} e esse provedor introduziu uma \textit{breaking change} na \textit{release}, por exemplo, \texttt{1.0.4}. Então foram instaladas as releases \texttt{p@1.0.2}, \texttt{p@1.0.3} e \texttt{p@1.0.5} para verificar se alguma dessas \textit{releases} não introduziu ou consertou a \textit{breaking change}. Assim, foi possível confirmar em qual \textit{release} do provedor a \textit{breaking change} foi introduzida/consertada.
\end{itemize}{}

Para todos os casos de erro confirmados como \textit{breaking changes}, foram coletadas a data das \textit{releases} dos provedores que introduziram as \textit{breaking changes} para realizar uma análise temporal. O registro do \textsf{npm} está funcional desde 2010 e foi analisada a evolução das \textit{breaking changes} ao longo do tempo.

Para entender quais características as \textit{releases} com \textit{breaking change} diferem das demais \textit{releases} sem \textit{breaking changes}, foi recuperado, para cada \textit{release} que introduziu a \textit{breaking change}, a quantidade de \textit{commits} que o provedor introduziu em todas as \textit{release} pertencentes ao mesmo nível \textit{major}. Por exemplo, para uma \textit{breaking change} introduzida no provedor \textsf{p@2.0.3}, foi recuperado a quantidade de \textit{commits} introduzida no \textit{range}  \textsf{p@2.x.y}, ou seja, \textsf{p@2.0.0}, \textsf{p@2.0.1}, \textsf{p@2.0.2}, \textsf{p@2.0.3} e assim por diante. Então, foi calculada a mediana dos \textit{commits} introduzidos em cada \textit{release} nesse \textit{range major} para verificar se a \textit{breaking change} na \textit{release} do provedor foi influenciada pela quantidade de \textit{commits}. Entretanto, três provedores foram removidos desta análise pois os seus repositórios são compartilhados com outros pacotes, o que tornou inviável analisar a quantidade de \textit{commits} entre duas \textit{releases}, uma vez que seria analisado \textit{commits} dos outros pacotes também. Esses provedores são \textsf{@types/node}, \textsf{@types/lodash} e \textsf{babel-preset-es2015}.

%---------------------------------------------------%
%----------------------RQ2--------------------------%
%---------------------------------------------------%

\section{QP2. Como os provedores introduzem \textit{breaking changes} em uma \textit{release}?}
\label{sec:qp2}

\subsubsection{Motivação}
\label{sec:qp2:motivation}
Pesquisas anteriores apresentam estudos sobre \textit{breaking changes} no ecossistema do \textsf{npm}. Entretanto, pelo fato do \textit{Javascript} ser dinâmico, esses estudos focaram apenas nas alterações de \textit{APIs}, tais como as remoções/renomeações, alterações na lista de parâmetros e alterações no tipo de retorno. Esses estudos foram realizados por  \citeonline{teorical_reference:bc_1} e \citeonline{noregrets2018} e não verificaram \textit{breaking changes} além das relacionadas às \textit{APIs}. Porém, podem haver outros tipos de \textit{breaking changes} no ecossistema do \textsf{npm} além das alterações em \textit{API}. Por causa da falta de informação, muitas \textit{breaking changes} são introduzidas, mas poderiam ser facilmente evitadas. Por isso, categorizar as \textit{breaking changes} ajudará os desenvolvedores a atentar-se para as \textit{breaking changes} mais comuns, assim produzindo códigos menos favoráveis às \textit{breaking changes}.

\subsubsection{Método}
\label{sec:qp2:approach}
O objetivo da análise manual é descobrir o motivo que originou uma \textit{breaking changes}, ou seja, qual foi a alteração que o provedor realizou que causou a \textit{breaking change}, para que seja possível agrupa-las por suas similaridades. Após descobrir qual versão de qual provedor a \textit{breaking change} foi introduzida, foi realiza uma análise manual no repositório do provedor para descobrir a exata alteração no código que originou a \textit{breaking change}. As próximas técnicas foram usadas:

\begin{itemize}
    \item \textbf{Arquivos de alterações:} os arquivos de registros de alterações, comumente nomeados por \textit{CHANGELOG.md} ou \textit{HISTORY.md}, contêm as descrições das principais alterações em cada \textit{releases} do projeto. Uma das informações mais relevantes nestes arquivos são as descrições de \textit{breaking changes}. Por exemplo, a versão \textit{5.0.0} do pacote \textsf{Mocha} contém uma \textit{breaking change} que foi documentada no \textit{CHANGELOG.md}\footnote{https://github.com/mochajs/mocha/blob/master/CHANGELOG.md\#500--2018-01-17} de acordo com a Figura \ref{fig:bc_documentation_mocha}. Outro tipo de documentação equivalente são as \textit{releases-notes}, como pode ser visualizado na Figura \ref{fig:bc_documentation_other} como o pacote \textsf{wpxml2md} documentou \textit{breaking changes} nas suas \textit{releases-notes}.\footnote{https://github.com/akabekobeko/npm-wpxml2md/releases/tag/v2.0.0}

    \item \textbf{Ferramentas de \textit{diff}:} foi utilizado ferramentas que realizam o  \textit{diff} entre duas \textit{releases} de um pacote. Um \textit{diff} entre duas \textit{releases} exibe todas as alterações que foram realizadas de uma \textit{release} para outra. Com isso, foi verificado o que foi adicionado e removido do código do provedor -- até mesmo do cliente -- em um determinado intervalo de versões.

    \item \textbf{\textit{Commits} dos provedores:} foi analisado os \textit{commits} do provedor que introduziu a \textit{breaking change} para verificar exatamente a sua evolução em detalhes. Foi verificado no repositório do provedor os \textit{commits} posterior e anterior ao \textit{commit} da \textit{release} com \textit{breaking change} para verificar exatamente em qual \textit{commit} a \textit{breaking change} foi introduzida.
\end{itemize}

\begin {figure} [h!]
   \centering
   \mbox {
        \subfigure[]{\label{fig:bc_documentation_mocha} \includegraphics[scale=0.5]{figuras/bc_documentation_mocha.pdf}}\quad
        \subfigure[]{\label{fig:bc_documentation_other} \includegraphics[scale=0.5]{figuras/bc_documentation_other.pdf}}
    }
    \caption{Exemplo de \textit{break changes} documentadas em \textit{changelogs} e \textit{relese-notes}}
    \label{fig:result_rq1_once_twice_three}
\end{figure}

Também foi buscado em \textit{issues} e \textit{pull-requests} por comentários indicando as causas das \textit{breaking changes}. Após descobrir as alterações que introduziram as \textit{breaking changes}, foram analisadas e agrupadas cada uma dessas alterações em categorias mais genéricas possíveis. Por exemplo, todas as alterações relacionadas com mudança no tipo de variáveis foram agrupadas em uma categoria chamada \textit{Alteração de tipo de objeto}. Também foi analisado o nível do Versionamento Semântico que a \textit{breaking change} foi introduzida pelo provedor e consertada pelo provedor ou cliente, bem como o local onde as \textit{breaking changes} foram documentadas (\textit{issues/pull-requests/changelogs}).

%---------------------------------------------------%
%----------------------RQ3--------------------------%
%---------------------------------------------------%

\section{QP3. Como os clientes se recuperam das \textit{breaking changes}?}
\label{sec:qp3}

\subsubsection{Motivação}
\label{sec:qp3:motivation}

Uma \textit{breaking change} pode impactar um pacote cliente através de uma atualização \textit{implícita} ou \textit{explícita} de seu pacote provedor. Uma atualização implícita ocorre quando o cliente especificou o seu provedor como um \textit{range} de versões no \textit{package.json}. Então o \textsf{npm} descarrega automaticamente a nova \textit{release} do provedor. Já uma atualização explícita ocorre quando o cliente atualiza manualmente a versão do provedor no \textit{package.json} e o \textsf{npm} descarrega a nova versão especificada pelo cliente. Após uma \textit{breaking change}, o cliente pode se recuperar realizando uma alteração no seu código, aguardando uma nova \textit{release} do provedor que venha a consertar a \textit{breaking change} ou o cliente pode realizar um \textit{downgrade/upgrade} na versão do provedor.

As \textit{breaking changes} podem ser introduzidas pelo provedor \textit{direto} ou pelo \textit{indireto}, uma vez que os clientes dependem de poucos provedores diretos mas dependem de muitos provedores indiretos \cite{npm-seven}. Mesmo quando o cliente tem poucos provedores diretos, muitos provedores indiretos podem propagar \textit{breaking changes}. Quando uma \textit{breaking change} se manifesta nos pacotes clientes, esses devem se recuperar uma vez que eles precisam executar sem erros e, também, eles podem ser provedores de outros pacotes nessa árvore de dependências. Portanto, uma \textit{breaking change} pode ser continuamente propagada enquanto não for consertada por nenhum dos pacotes. Até mesmo quando as \textit{breaking changes} podem ser consertadas atualizando para uma nova versão do provedor, os pacotes clientes precisam resolver manualmente as incompatibilidades que ainda existem \cite{Foo:2018:ESC:3236024.3275535}. Então, entender o comportamento da manifestação das \textit{breaking changes} pode ajudar os desenvolvedores a compreenderem quais são as maneiras mais efetivas e rápidas para se recuperar das \textit{breaking changes}.

\subsubsection{Método}
\label{sec:qp3:approach}
Todas as informações usadas para responder esta questão de pesquisa foram recuperadas dos repositórios dos pacotes clientes. Foi procurado nesses repositórios por informações sobre o erro e como os cliente se recuperaram da \textit{breaking change}. As seguintes informações foram analisadas:

\begin{itemize}
    \item \textbf{\textit{Commits:}} foi analisado manualmente os próximos \textit{commits} no repositório do cliente a partir da data da \textit{release} que contém a \textit{breaking change}. Foram analisados principalmente os \textit{commits} que alteraram o \textit{package.json} para verificar se o cliente realizou um \textit{downgrade/upgrade} ou se o cliente removeu ou substituiu o provedor.

    \item \textbf{\textit{Changelogs:}} o cliente pode mencionar nos \textit{changelogs} e nas \textit{release-notes} como foi realizada a recuperação da \textit{breaking change}, principalmente se o cliente realizou um \textit{downgrade/upgrade} na versão do provedor. Também, se o cliente consertou a \textit{breaking change} diretamente no seu código, provavelmente há essa informação nos \textit{changelogs}. Ao todo, 48\% dos repositórios dos clientes continham \textit{changelog} ou \textit{release-notes}.
    
    \item \textbf{\textit{Pull-requests/Issues:}} foi procurado por \textit{pull-requests} e \textit{issues} no repositório do cliente que deveria consertar, ou conter informações sobre a \textit{breaking change}. \textit{Pull-requests/issues} nomeadas como \textit{Update provider}, \textit{Fix provider errors}, \textit{Fix tests} indicavam a presença de alguma alteração que foi realizada devido às \textit{breaking changes}.
\end{itemize}

Para cada caso de \textit{breaking change} foi recuperado a árvore de dependências do cliente até o provedor que introduziu a \textit{breaking change}. Por exemplo, em nosso segundo exemplo motivacional (Capítulo \ref{cap:exemplos}) foi recuperado a árvore de dependências a partir do cliente até \textsf{broccoli-asset-rev$\rightarrow$broccoli-filter$\rightarrow$broccoli-plugin} (Figura \ref{fig:dependency_tree}). Com isso foi analisada a quantidade de \textit{breaking changes} introduzida por provedores diretos e indiretos. 

Também foi investigado os dados sobre quando a \textit{breaking change} foi introduzida, consertada, qual pacote consertou e como foi realizada a correção. Assim foi analisado o tempo que as \textit{breaking changes} levaram para serem consertadas e quais são as principais maneiras com que os clientes se recuperam das \textit{breaking changes}. Ainda foi analisado os casos onde o provedor consertou a \textit{breaking change} e, ainda assim, o cliente realizou um \textit{upgrade/downgrade} da versão do provedor. Por fim, foi verificado como a versão dos provedores foram alteradas pelos clientes e como a documentação da \textit{breaking change} influenciou na velocidade com que as \textit{breaking changes} foram consertadas. % Esse capítulo e nome é apenas uma sugestão.
\chapter{Coleta de Dados}
\label{cap:metodologia}

O objetivo deste trabalho é analisar as \textit{breaking changes} em \textit{releases minor} e \textit{patch} no ecossistema do \textsf{npm}. Para realizar isso, de um conjunto de dados sobre os pacotes do \textsf{npm}, descrito na Seção \ref{sec:col_base}, foi utilizada uma amostra aleatória representativa, conforme explicado na Seção \ref{sec:col_amostra}. Dessa amostra, cada um dos repositórios dos pacotes clientes foram copiados localmente e todas as suas \textit{releases} que continham alterações de provedores tiveram seus testes executados, para detectar se continham algum erro, conforme a Seção \ref{sec:bcdetect} explica. Após executarem, os clientes e \textit{releases} que resultaram em algum erro foram analisados e classificados em \textit{erros do cliente}, \textit{breaking changes}, \textit{breaking without-change}, e \textit{erros não identificados}, conforme a Seção \ref{sec:col_dados}.

\section{Coleta do Conjunto de Dados}
\label{sec:col_base}
O conjunto de dados utilizado neste trabalho foi extraído do registro do \textsf{npm} do qual foram recuperados os arquivos metadados \textit{package.json} de 1,233,944 pacotes publicados no período de 20 de Dezembro de 2010 até 01 de Abril de 2020. Os principais dados recuperados no \textit{package.json} são os \textit{timestamp} de cada uma das \textit{releases} dos pacotes, os provedores que os pacotes clientes continham em cada \textit{release} e suas respectivas versões. A Figura \ref{fig:package_json} exibe as informações do pacote \textsf{buffer-includes}\footnote{http://registry.npmjs.org/buffer-includes} que podem ser recuperadas de seu \textit{package.json}.

\begin{figure}
    \centering
    \includegraphics[scale=0.65]{figuras/package_json.pdf}
    \caption{Informações que serão recuperadas do \textit{package.json} para validar um pacote.}
    \label{fig:package_json}
\end{figure}{}

Foram excluídos desse conjunto de dados todos os pacotes que não continham nenhum provedor, pois quando um pacote não contém provedores não há como ser impactado por \textit{breaking changes}. Dessa maneira, o conjunto de dados final foi reduzido para um total de 987,595 pacotes clientes do \textsf{npm}. Por fim, para cada \textit{release} de todos os clientes, foram resolvidas as versões de todos os provedores com base no \textit{timestamp} dessa \textit{release}. Ou seja, foi resolvido o \textit{range} do provedor, que o cliente especificou naquela \textit{release}, para a maior versão aceita no momento da \textit{release} do cliente. Desse modo, foi determinado qual a versão do provedor que o cliente utilizava no momento da publicação da \textit{release}.

\section{Amostragem}
\label{sec:col_amostra}
% para remover a lista de siglas, foi utilizado um \textit no http
Para este trabalho, foi utilizada uma amostra do conjunto de dados especificado na Seção \ref{sec:col_base}, que contém um total de \textit{987k} clientes e, devido ao à sua dimensão, não é viável realizar uma análise manual para essa quantidade de clientes. Por isso, foi utilizada uma amostra representativa contendo 384 clientes, com base em um cálculo amostral com 95\% de confiança e $\pm$5\% de margem de erro. Os 384 clientes foram recuperados aleatoriamente e verificados, a cada pacote sorteado, se cumpre três requisitos: 1) possuir um \textit{script} de teste não vazio e diferente do \textit{script} padrão de teste do \textsf{npm}: \texttt{\{"test": "Error: no test specified"\}} (488,805 conferem); possuir a \textit{url} do repositório (410,433 conferem); e o repositório precisa existir -- foi esperado através de uma requisição \textit{HyperText Transferer Protocol} (HTTP) para o repositório o código \textit{200} indicando sua existência. Todos esses três requisitos foram analisados na última \textit{release} disponível do pacote.

Foi realizada uma verificação manual nos repositórios do pacotes com menos de 4 \textit{releases} para verificar se o pacote não era um \textit{toy package}, ou seja, um pacote que não foi criado para ser um projeto real, apenas um teste no \textsf{npm}, no \textsf{GitHub} ou algo do tipo. Então, dos 384 clientes que tiveram seus testes executados, em 34 não foi possível executar o comando \texttt{npm install}/\texttt{test} para nenhuma de suas \textit{releases}. Desses 34 clientes, 15 não possuíam algum dos arquivos necessários para os testes; 11 continham \textit{scripts} de teste inválidos em todas as suas \textit{releases}, tal como \texttt{\{"test": "no tests"\}}; 4 haviam listados alguns dos arquivos no \textit{.gitignore} -- arquivo utilizado pelo \textsf{git} para ignorar arquivos no repositório --, mas que eram necessários para a execução dos testes; 2 necessitaram de configurações específicas em banco de dados e não foi possível realizá-las; 1 cliente foi considerado como um \textit{toy package}, ou seja, não era um projeto real, mas apenas um repositório com um único arquivo; e 1 cliente requeria uma variável de ambiente para acessar um determinado site. Dessa forma, os 34 clientes foram substituídos em um novo sorteio seguindo o mesmo critério do sorteio anterior, totalizando 384 clientes com 5957 \textit{releases} que foram utilizados no estudo. Um detalhe importante refere-se aos clientes que utilizavam algum tipo de sistemas de banco de dados como o \textsf{MySql}. Quando um erro foi ocasionado pela falta de uma configuração básica, tal como executar um \textit{script} para criar uma tabela, então essas configurações foram realizadas e os testes do cliente foram re-executados. Somente quando o cliente necessitava de uma configuração específica e que não foi possível realizá-la, o cliente foi substituído por outro.

\section{Detecção de \textit{Breaking Changes}}
\label{sec:bcdetect}
Para este trabalho, foi desenvolvida uma ferramenta chamada \textsf{BCDetect}\footnote{https://github.com/danielventurini/bcdetect} disponível no \textsf{GitHub} sob a licença \textsf{MIT}. Esta ferramenta clona o repositório do respectivo cliente -- todos os clientes estavam hospedados no \textsf{Github} -- e cria uma estrutura de dados para armazenar as informações sobre o cliente. Nessa estrutura cada \textit{release} do cliente contém todos os provedores com suas versões resolvidas e tipo de atualização que os provedores realizaram desde a última \textit{release} do cliente: \texttt{steady} significa que o provedor não publicou nenhuma \textit{release} aceita desde a última release do cliente; \texttt{upgrade} significa que o provedor publicou uma nova \textit{release} aceita desde a última \textit{release} do cliente; e quando não há nenhuma dessas informações, o provedor foi inserido no \textit{package.json} nesta \textit{release}. Essa estrutura básica está representada na Figura \ref{fig:bc_work}, que seria construída a partir dos dados do cliente \textsf{buffer-includes} da Figura \ref{fig:package_json}.

\begin{figure}
    \centering
    \includegraphics[scale=0.9]{figuras/bcdetect_work.pdf}
    \caption{Estrutura de dados para representar o cliente \textsf{buffer-includes}.}
    \label{fig:bc_work}
\end{figure}{}

Os testes de cada \textit{release} do cliente foram executados toda vez que havia pelo menos um provedor com nova \textit{release} (\textit{upgrade}) ou um provedor havia sido adicionado naquela \textit{release} do cliente. Após clonado o repositório, foi executado o comando \texttt{git checkout} para todas as \textit{releases} do cliente -- \textit{release} por \textit{release} -- com a respectiva \textit{tag} da \textit{release}, fazendo com que todos os arquivos sejam restaurados para exatamente os mesmos arquivos do momento em que o cliente publicou a \textit{release}. Uma \textit{tag} é uma referência a um ponto importante e específico do repositório, que geralmente são as \textit{releases}. Para \textit{releases} que o desenvolvedor não criou uma \textit{tag}, o \textit{checkout} foi realizado usando o \textit{timestamp} da respectiva \textit{release}. Após, o arquivo \textit{package-lock.json}\footnote{https://docs.npmjs.com/files/package-lock.json} foi excluído, pois esse arquivo altera o comportamento do comando \texttt{npm install} -- a partir do \textsf{npm@5} -- fazendo com que o \textsf{npm} instale as versões dos provedores de acordo com o \textit{package-lock.json}, e não de acordo com o \textit{package.json}. Em seguida, todos provedores no \textit{package.json} e suas respectivas versões foram adicionados apenas no campo \textit{dependencies} do \textit{package.json} e os demais campos foram removidos,\footnote{campos para dependências no \textit{package.json}, tais como o \textit{peerDependencies}, \textit{optionalDependencies} e o \textit{globalDependencies}} uma vez que para executar os testes, ambos os provedores são requeridos.

O \textsf{Node.js} é o ambiente de execução para os pacotes \textsf{JavaScript} e a cada 6 meses uma nova \textit{release major} é publicada.\footnote{https://github.com/nodejs/node\#release-types} Por isso, antes de executar os testes da \textit{release} do cliente, a versão do \textsf{Node.js} precisa ser alterada. O chaveamento das \textit{releases} do \textsf{Node.js} é necessário pois as \textit{releases major} do \textsf{Node.js} não são retro-compatíveis, ou seja, um pacote que executa com sucesso na versão \textit{0.x} do \textsf{Node.js}, por exemplo, provavelmente não executaria com sucesso na versão \textit{8.x}. Para cada \textit{release} do cliente que teria seus testes executados, a versão do \textsf{Node.js} foi selecionada de duas maneiras: 1) do campo \texttt{engines->node} no \textit{package.json}, que permite o desenvolvedor especificar a versão do \textsf{Node.js}; e 2) atravéś do \textit{timestamp} da \textit{release} do cliente, foi possível identificar a última \textit{release} do \textsf{Node.js} disponível,\footnote{https://nodejs.org/en/download/releases} ou seja, qual era a \textit{release} máxima do \textsf{Node.js} que os testes dos cliente foram executados no momento da \textit{release} do cliente. Assim, os testes do cliente foram executados em todas as versões \textit{major} do \textsf{Node.js}, da versão mais atual, pelo \textit{timestamp} da \textit{release} do cliente, até a versão \textit{major} mais antiga, ou até o teste ocorrer com sucesso. Para o cliente da Figura \ref{fig:bc_work}, a sua \textit{release 0.1.0} possui o \textit{timestamp} como \textit{2015-10-28}, e a última \textit{release} do \textsf{Node.js} disponível até esta data é a \textit{4.2.1}. Assim, os testes dessa \textit{release} do cliente seriam executados com as \textit{releases 4.x, 3.x, 2.x, 1.x, 0.x} do \textsf{Node.js}. Ao atualizar a \textit{release} do \textsf{Node.js}, a versão do \textsf{npm} é atualizada também. Isso é necessário pois é o \textsf{npm} que executa os \textit{scripts install} e \textit{test}. Após executar o \textit{install} e o \textit{test}, foi salvo as seguintes informações:

\begin{itemize}
    \item versão do cliente;
    \item se houve alteração na versão aceita de alguns dos provedores;
    \item os códigos da execução do \texttt{npm install} e \texttt{npm test} -- sucesso ou erro;
    \item a versão do \textsf{Node.js} que deveria ser executado com base na data da \textit{release}; e
    \item a versão do \textsf{Node.js} que os testes do cliente executaram com sucesso.
\end{itemize}{}

Os passos resumidos das operações em cada cliente juntamente com a verificação de alteração dos provedores e execução dos testes das \textit{releases} se encontram na Figura \ref{fig:steps_work}.

\begin{figure}
    \centering
    \includegraphics[scale=0.7]{figuras/steps_work.pdf}
    \caption{Etapas para clonar, restaurar e execução os testes das \textit{releases} de um cliente}
    \label{fig:steps_work}
\end{figure}{}

\subsection{Resultados da Execução dos Testes dos Clientes}
\label{sec:col_dados}

Ao todo, 384 clientes foram utilizados neste trabalho e foram executadas os comandos \texttt{npm install/test} em pelo menos uma de suas \textit{releases}. Desses 384 clientes, 203 resultaram em erros para alguma de suas \textit{releases}. Analisando os resultados por \textit{releases}, de todas as 5957 \textit{releases}, foram executadas os testes de um total de 3230 \textit{releases}, enquanto que 2727 \textit{releases} não tiveram seus testes executados pois não haviam alterações nas \textit{releases} dos provedores, ou não continham algum \textit{script} de teste válido. Após o término da execução, 1954 \textit{releases} resultaram em sucesso, enquanto que 1276 \textit{releases} resultaram em erros no \textit{script install/test}. A Tabela \ref{tab:res_rq1_1} contém os resultados prévios sem que os clientes que resultaram em erro fossem analisados, ou seja, são dados apenas da execução dos testes dos clientes.

\begin{table}[]
\centering
\begin{tabular}{lrr}
\toprule
                    & Clientes & \textit{Releases} \\ \hline
    Total           & 384     & 5957     \\
    Não executado   & 0       & 2727     \\
    Executado       & 384     & 3230     \\
    Sucesso         & 181     & 1954     \\
    Erro            & 203     & 1276     \\ \bottomrule
\end{tabular}

\caption{Resultado da execução dos testes nas \textit{releases} de cada cliente, por clientes e por \textit{releases}.}
\label{tab:res_rq1_1}
\end{table}

% Após executarem, todas as 1276 \textit{releases} que resultaram em erros foram analisadas manualmente para separar os erros falso-positivos dos erros reais. Conforme explicado na Seção \ref{sec:qp1:approach}, os erros do tipo falso-positivos são erros que foram gerados pela falta de alguma configuração, isso é, devido uma má configuração, a execução dos testes da \textit{release} resultou em erros. Os erros falso-positivos mais comuns estavam relacionados a serviços que necessitavam de configurações prévias, tais como o \textsf{mysql} e o \textsf{mongodb}, que por vezes necessitavam de tabelas, senhas, \textit{scripts}, entre outras configurações para que os clientes executassem com sucesso seus testes. Após as \textit{releases} serem analisadas manualmente, foi constatado um total de 38 clientes com falso-positivos em todas as suas \textit{releases}, totalizando 172 \textit{releases} e mais 244 \textit{releases} que impactaram parcialmente outros clientes, ou seja, não foram o único tipo de erro no clientes. Assim, todos os 38 clientes e as 416 \textit{releases} foram consideradas como se os testes estivessem executados com sucesso. Dessa maneira, após a análise prévia, os dados atualizados são mostrados na Tabela \ref{tab:res_rq1_2} e na Figura \ref{fig:res_rq1_g}.

%\begin{table}[]
%\centering
%\begin{tabular}{lrr}
%\toprule
%                    & Clientes & \textit{Releases} \\ \hline
%    Sucesso         & 191     & 1388     \\
%    Erro            & 193     & 912     \\ \bottomrule
%\end{tabular}
%\caption{Resultado da execução dos testes, contabilizando os falso-positivos, por clientes e \textit{releases}.}
%\label{tab:res_rq1_2}
%\end{table}

%\begin{figure}
%    \centering
%    \includegraphics[scale=0.5]{figuras/general_results.pdf}
%    \caption{Resultado da execução dos testes, por clientes e por \textit{releases}.}
%    \label{fig:res_rq1_g}
%\end{figure}{}
 % Esse capítulo e nome é apenas uma sugestão.
\chapter{Resultados Preliminares}
\label{cap:res_pre}

Neste capítulo encontram-se os resultados preliminares da primeira e da segunda questão de pesquisa.

\section{QP1. Com que frequência \textit{breaking changes} impactam os clientes}
\label{res:qp1}

\subsubsection{\textbf{10.1\% dos clientes e 8.1\% das \textit{releases} sofreram \textit{breaking changes}}}

Após os clientes/\textit{releases} executarem, os que geraram erros foram analisados para se confirmar a origem do erro: uma chamada à uma função do provedor que contém uma \textit{breaking change} ou alguma alteração realizada pelo cliente. Do total de 184 clientes com erro, 96 sofreram casos de erros internos, enquanto que 45 sofreram algum dos casos particulares de \textit{breaking change}. Por fim, 39 clientes sofreram \textit{breaking changes} em uma de suas \textit{releases}. Também, em 31 clientes houve alguma \textit{release} da qual não foi encontrado o motivo do erro. Porém, um cliente que sofreu uma \textit{breaking change}, por exemplo, pode ter sofrido também com erros internos, e vice-versa, pois um caso não influência na ocorrência dos demais. Por isso, os resultados são melhores apresentados em função das \textit{releases}, uma vez que as \textit{releases} só podem sofrer com apenas um tipo de erro. Dessa maneira, do total de 907 \textit{releases} que sofreram algum erro, foram identificadas 431 \textit{releases} com erros internos, 213 \textit{releases} com erros dos casos particulares de \textit{breaking changes}, 190 erros do caso de \textit{breaking changes} e em 73 \textit{releases} não foi possível descobrir o motivo que gerou o erro. A Figura \ref{fig:pre_res_rq1} contém os resultados em função das \textit{releases}. Assim, os dados sobre os clientes e \textit{releases} que sofreram com \textit{breaking changes} serão utilizados para responder as demais questões de pesquisa.

\begin{figure}
    \centering
    \includegraphics[scale=0.7]{figuras/pre_res_rq1.pdf}
    \caption{Resultado dos casos de erros em função das \textit{releases}}
    \label{fig:pre_res_rq1}
\end{figure}{}

\section{QP2. Como os provedores introduzem \textit{breaking changes} em uma \textit{release}}
\label{res:qp2}

\subsubsection{\textbf{Classificação das \textit{Breaking changes}}}

Ao todo, foram 43 casos de \textit{breaking changes} distribuídas em 39 clientes. Todos esses casos foram agrupados em 8 diferentes categorias das quais se encaixavam. A Tabela \ref{tab:bc_category} apresenta cada uma dessas categorias, bem como a quantidade de clientes e a quantidade de \textit{releases} que cada categoria atingiu.

\begin{table}[]
\begin{tabular}{|l|c|c|c|c|}
\hline
\centering
\textbf{Categoria}           & \textbf{Pacotes afetados} & \textbf{\%}   & \textbf{\textit{Release} afetadas} & \textbf{\%}    \\ \hline
Alteração de regras          & 12              & 27,9 & 64                          & 33,68 \\
Provedores incompatíveis     & 8               & 18,6 & 30                          & 15,78 \\
Alteração de tipo de objeto  & 8               & 18,6 & 24                          & 12,63 \\
Objeto indefinido            & 4               & 9,3  & 25                          & 13,15 \\
Código errado                & 4               & 9,3  & 13                          & 6,84  \\
Código não-atualizado        & 3               & 6,97 & 25                          & 13,15  \\
Renomeação de função         & 3               & 6,97 & 5                           & 2,63  \\
Arquivo não encontrado       & 1               & 2,32 & 4                           & 2,1  \\ \hline
\textbf{Total}               & 43              &      & 190                         &       \\ \hline
\end{tabular}
\caption{Categorias dos casos de \textit{breaking change}}
\label{tab:bc_category}
\end{table}

A seguir, encontra-se uma descrição sobre cada categoria.

\begin{itemize}
    \item \textbf{Alteração de regras}: este caso foi o principal que impactou os clientes. Essa categoria contém os casos de \textit{breaking change} no qual os provedores possuíam um determinado comportamento, mas alteraram algumas de suas regras/funcionalidades e impactaram os seus clientes. Não foi uma simples alteração no código, tal como uma alteração de tipo de variáveis, ou um código escrito de maneira errada, mas sim uma regra no qual o cliente tinha como sólida, foi alterada; %Por exemplo, o pacote \textit{request@2.18.0} introduziu uma alteração em seu código\footnote{https://github.com/request/request/commit/d05b6ba72702c2411b4627d4d89190a5f2aba562\#diff-168726dbe96b3ce427e7fedce31bb0bcR857}, como pode ser visto na Figura \ref{fig:bc_category_change_rule_1}.

    %\begin{figure} \centering \includegraphics[scale=0.6]{figuras/bc_category_change_rule_1.png} \caption{Alteração de regra de funcionamento do \textit{request}} \label{fig:bc_category_change_rule_1} \end{figure}{}

    %Nesse caso, o \textit{request} adiciona uma \textit{string} vazia ao invés de manter \textit{undefined} o corpo de uma requisição. Esse caso do \textit{request} ocorreu exatamente como foi explicado por \citeonline{Foo:2018:ESC:3236024.3275535} dizendo que os pacotes evoluem independentemente dos clientes. Essa alteração na regra do \textit{request} reflete em uma evolução do pacote, mas o cliente não esperava essa alteração e confiava que o corpo da resposta fosse retornado como \textit{undefined} em caso de erro, por isso o cliente quebrou.

    \item \textbf{Provedores incompatíveis}: nessa categoria, há um provedor direto A e um provedor indireto B envolvido, o qual alterou o seu código, o que não gerou um erro, mas provocou no provedor A um comportamento inesperado, ou seja, o provedor B passou a ser incompatível com o provedor A. Nessa categoria, nenhum dos provedores contém um erro, mas sim uma incompatibilidade; %Um exemplo disso ocorreu com os pacotes \textit{babel-eslint}\footnote{https://www.npmjs.com/package/babel-eslint} e \textit{escope}\footnote{https://www.npmjs.com/package/escope}, entretanto, o pacote \textit{escope} é um provedor indireto do \textit{babel-eslint}.

    %\begin{figure} \centering \includegraphics[scale=0.5]{figuras/bc_category_incompatibles_providers.png} \caption{Alteração de código do \textit{escope}} \label{fig:bc_category_incompatibles_providers} \end{figure}{}

    %A \textit{releases escope@3.4} realizou uma alteração no seu código, de acordo com a Figura \ref{fig:bc_category_incompatibles_providers}, mas que não reflete em um erro. Essa alteração impactou diretamente o pacote \textit{babel-eslint}, mesmo o pacote \textit{escope} não sendo um provedor direto do \textit{babel-eslint} e não ter introduzido um erro\footnote{https://github.com/estools/escope/issues/99\#issuecomment-178151491}. Com isso, há uma incompatibilidade entre os provedores e essa incompatibilidade precisou ser corrigida pelo \textit{babel-eslint} e não pelo \textit{escope}. Essa foi a \textit{breaking change} que mais surgiu na análise manual pois, dos 43 casos, 5 (11.6\%) refletiam essa incompatibilidade, uma vez que o \textit{babel-eslint} é provedor de 5.8\% de toda a base de dados.

    \item \textbf{Alteração de tipo de objeto}: essa é uma categoria de \textit{breaking changes} facilmente detectável em linguagens fortemente tipadas, mas no \textit{Javascript} representam um tipo de \textit{breaking change} que, por muitas vezes, pode nem afetar o código do cliente. Mas, neste trabalho, foram detectados 8 (18.6\%) de casos nos quais os provedores alteraram o tipo de alguma variável;

    %\begin{figure} \centering \includegraphics[scale=0.5]{figuras/bc_category_change_type.png} \caption{Alteração de um tipo \textit{array} para \textit{object}} \label{fig:bc_category_change_type} \end{figure}{}

    %Na Figura \ref{fig:bc_category_change_type} o provedor \textit{socket.io}\footnote{https://www.npmjs.com/package/socket.io} alterou alguns \textit{arrays} para \textit{object}\footnote{https://github.com/socketio/socket.io/commit/b73d9bea4efb48277eee685763026ff2df5a79ab}. Anteriormente, os clientes iteravam nesses \textit{arrays}, mas após essa alteração, os clientes foram afetados.

    \item \textbf{Objeto indefinido}: por vezes, os códigos podem estar todos corretos, mas então o provedor tenta acessar uma variável que não existe. Essa categoria de \textit{breaking change} representa os casos no qual os provedores tentaram obter acesso à alguma variável/objeto, mas que não existiam. Esses erros são os que facilmente podem ser consertados/evitados apenas adicionando o código da Listagem \ref{cod:undefined_object}:

    \begin{lstlisting}[style=bash, label=cod:undefined_object]
    this.var = this.var || {};
    \end{lstlisting}

    %Esse tipo de erro surgiu no pacote \textit{ember-cli-htmlbars-inline-precompile}\footnote{https://www.npmjs.com/package/ember-cli-htmlbars-inline-precompile}, no qual o desenvolvedor tenta acessar uma variável que não estava disponível. Mas, assim como o desenvolvedor já havia feito com as demais variáveis da Figura \ref{fig:bc_category_undefined_object}, uma simples alteração no código foi o suficiente.

    %\begin{figure} \centering \includegraphics[scale=0.7]{figuras/bc_category_undefined_object.png} \caption{Correção do erro de objeto indefinido} \label{fig:bc_category_undefined_object} \end{figure}{}

    \item \textbf{Código errado}: este caso de \textit{breaking change} ocorreu quando o provedor escreveu um código semanticamente incorreto, gerando um erro na sua execução e afetando o cliente. Em linguagens compilada, esse tipo de erro seria facilmente identificado pelo compilador em tempo de compilação; %Foi exatamente isso que a dependência fez. Ao alterar o seu código, o desenvolvedor escreveu duas vezes a mesma variável, como pode ser visto na Figura \ref{fig:bc_category_wrong_code}. Assim como os erros do tipo \textit{undefined object}, os erros dessa categoria  são facilmente corrigidos.

    %\begin{figure} \centering \includegraphics[scale=0.8]{figuras/bc_category_wrong_code.png} \caption{Código semanticamente incorreto} \label{fig:bc_category_wrong_code} \end{figure}{}

    \item \textbf{Código não-atualizado}: as \textit{breaking changes} desta categoria são as que o provedor atualiza o \textit{range} de seu provedor mas não altera o seu código para adaptar-se a ele. Assim, o erro está no primeiro provedor que contém um código desatualizado;

    \item \textbf{Renomeação de função}: as \textit{breaking changes} relacionadas à esta categoria foram facilmente detectáveis. Quando a mensagem de erro do \textit{node.js} era exibida como \textit{TypeError: var is not a function}, com pouca investigação já era possível identificar que uma determinada função não estava mais disponível, ou seja, havia sido removida ou alterado o seu nome; e

    %\begin{figure} \centering \includegraphics[scale=0.6]{figuras/bc_category_renamed_function.png} \caption{Alteração do nome de função} \label{fig:bc_category_renamed_function} \end{figure}{}

    \item \textbf{Arquivo não encontrado}: os casos de \textit{breaking change} relacionados à esta categoria são aqueles no qual o desenvolvedor realiza um acesso a um arquivo, mas esse não existe. O arquivo requerido pode não existir ou não estar disponível, uma vez que, referenciado no arquivo \textit{.npmignore} -- arquivo utilizado pelo \textit{npm} para ignorar arquivos durante o processo de publicação --, o arquivo existe mas não está disponível, mas também o arquivo pode não existir. Entretanto, o único caso de arquivo não encontrado ocorreu pois o arquivo \textit{index.js} estava referenciado no \textit{.gitignore}.% O provedor \textit{esprima-extract-comments}\footnote{https://www.npmjs.com/package/esprima-extract-comments} utilizava como provedor um \textit{fork} do pacote \textit{esprima}\footnote{https://github.com/ariya/esprima/} e o referencia em seu  \textit{package.json} para ser descarregado diretamente do \textit{Github}\footnote{https://github.com/jonschlinkert/esprima-extract-comments/blob/6b65a0f52f85bc6fa830d44e352ec3da9e9ef620/package.json\#L47}. Entretanto, o \textit{index.js} desse \textit{fork}, foi referenciado no \textit{.gitignore} e não estava disponível quando o \textit{npm} descarregou o pacote diretamente do \textit{Github}, mas o arquivo estava disponível se o pacote \textit{exprima} fosse descarregado diretamente do \textit{npm}.

\end{itemize}{} % Esse capítulo e nome é apenas uma sugestão.
\chapter{Cronograma de Atividades}
\label{sec:cronograma}

Nesta seção são apresentadas as atividades a serem desenvolvidas para a execução da proposta. O cronograma de realização das tarefas é apresentado na Tabela~\ref{tab:cronograma}.

\begin{enumerate}
\item \textbf{Documentação da Ferramenta.}
\item \textbf{Obtenção dos dados da RQ3.}
\item \textbf{Análise dos Resultados.}
\item \textbf{Escrita do TCC 2.}
\item \textbf{Entrega do TCC 2.}
\item \textbf{Apresentação do TCC 2.}
\end{enumerate}

\begin{table}[h!]
\centering
\renewcommand{\arraystretch}{1.3}
\caption{Cronograma de atividades}
\label{tab:cronograma}
\scalefont{0.9}
\begin{tabular}{|c|c|c|c|c|c|}
\hline
\multirow{2}{*}{\textbf{Atividade}} & \multicolumn{2}{l|}{\textbf{2019}} & \multicolumn{3}{l|}{\textbf{2020}} \\ \cline{2-6} 
             & Nov & Dez & Jan & Fev & Mar \\ \hline
\textbf{1}   &  X  &     &     &     &     \\ \hline
\textbf{2}   &  X  &  X  &     &     &     \\ \hline
\textbf{3}   &  X  &  X  &     &     &     \\ \hline
\textbf{4}   &     &  X  &  X  &  X  &     \\ \hline
\textbf{5}   &     &     &     &     &  X  \\ \hline
\textbf{6}   &     &     &     &     &  X  \\ \hline
\end{tabular}
\end{table} % Esse capítulo e nome é apenas uma sugestão.

% Apendices.
%\appendix
%\chapter{Instalação de Ferramentas}
\label{ape:instalacao:ferramentas}

Os apêndices são usados para disponibilizar materiais extras que por questões de espaço ou estilo de escrita não foram colocados diretamente no texto. Por exemplo, \textit{scripts}, instruções de instalação das ferramentas utilizadas pelo trabalho, partes de código fonte e questionários que tenham sido aplicados, tabelas com resultados...

(ATENÇÃO - veja com o seu orientador se é necessário disponibilizar algum material extra sobre algum capítulo em anexo!)



%bibliografia
\bibliographystyle{abntex2-alf}
\bibliography{main} % geração automática das referências a partir do arquivo main.bib

\backmatter
\end{document}
