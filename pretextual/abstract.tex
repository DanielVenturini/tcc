% Caso seja TCC 2, precisa traduzir o resumo e as palavras-chaves para inglês:
\begin{abstract}

\textbf{Context:} Packages hosted on \textsf{npm} create a dependency network, where \textit{client} packages are the ones that depends on \textit{provider} packages. Occasionally, providers introduce breaking changes, which are changes that may cause defects on clients. These changes should be only introduced in \textit{major} level of Semantic Versioning, but when introduced in \textit{minor} or \textit{patch} levels, these may cause issues on clients. \textbf{Objective:} This work proposes a study about breaking changes in minor and patch levels on \textsf{npm}. Our objectives are: 1) to measure and 2) to categorize the breaking changes, and 3) to analyze how clients recover themselves. \textbf{Method:} From a sample of clients from \textsf{npm}, we restored the releases and installed the latest version of providers that the client accepted in the release timestamp. Following, we executed the \texttt{npm install/test} scripts. All releases that raised an error were analyzed, and the client and providers code and repositories was verified to check whether the error was raised by a provider, characterizing a breaking change. \textbf{Results:} Altogether, 55 providers introduced breaking changes that impacted 13.9\% of client releases and these breaking changes have increased 63.4\% from the respective previous year. Yet, 54.9\% of provider releases with breaking changes have more commits than their other releases. Breaking changes are introduced in minor and patch level in the same proportion, but the majority is fixed by providers in patch levels and are documented in 78.1\% of cases, mainly on issues, causing the fix to be 3.3 time faster. While indirect providers are the ones that introduces the majority of breaking changes, clients fix these in 39.1\% of cases and they prefer to do an upgrade on the provider's version without changing the range. \textbf{Conclusions:} Breaking changes do really happen in minor and patch releases. Previous studies focused only on API breaking changes, while this study used clients' tests to find any types of breaking changes. We presented several suggestions to developers to improve their interaction with the \textsf{npm} ecosystem.

% Palavras-chaves em inglês, separadas por ponto.
\keywords{\textsf{npm}. Breaking change. Semantic Version. Dependency management.}
\end{abstract}