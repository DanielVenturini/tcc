% Caso seja TCC 2, precisa traduzir o resumo e as palavras-chaves para inglês:
\begin{abstract}
The npm is the largest package manager. The packages hosted in npm may depend on each other, making a dependency network between them, where the \textit{client} package can reuse several interfaces from a \textit{provider} package. The packages often update themselves publishing releases that contain new functionalities, bug fix and security updates. However, the provider packages evolve regardless of their clients, and these new releases may contain some errors and may cause unexpected behavior in client packages after an update. This unexpected behavior is called \textit{breaking change}. In this paper, we investigate breaking changes in the npm ecosystem. For this, we selected randomly a sample of 384 client packages from the npm registry, restored to a release and downgraded the providers for the latest accepted version in that release, executed the test scripts and saved the result. So, we address three research questions. In our first RQ, we measure the extent to which breaking changes affect the client package. Our manual analysis of the result tests showed that 10.4\% of the packages and 8.4\% of releases have undergone breaking changes. Also, the occurance of breaking changes in client packages is more related to the frequency of publication providers releases than to the number of providers the client packages have. For the last, 38.2\% of providers introduce breaking changes after 75\% of the development stage. Our second RQ is focused on which issues the provider packages cause the manifestation of a breaking change. From all breaking changes cases, we analyzed the provider to discover which issue raised the breaking change and we grouped them into some categories (e.g., \textit{change rules}, \textit{incompatible providers}, \textit{change the object types}, and so on) which ones indicate the most common issues and it helps the developers to avoid these. Further, we observed that 57.8\% of breaking changes are introduced into a minor release, and 76.9\% of these breaking changes are fixed by client packages into a patch release. This result shows that unexpected behavior is not always breaking changes, but those were a new functionality that the client packages had to adapt, for this, the client packages fixed their codes into a patch release. For the last, we did a relationship between the categories and the Semantic Version level providing information about breaking changes characteristics in each level. In our last RQ, we analyzed how client packages recover from the manifestation of breaking changes. We observed that the client packages recover from breaking change in 66.7\% of cases and their new releases are published, in median, 4 days after a breaking change manifestation, against 35 days for provider packages. Also, we analyzed that 82.2\% of breaking changes are implicit, that is, caused by a range version, and the client packages recover from breaking changes, in 66.7\%, just doing a downgrade in provider's version. This observation shows to developers which is the most common and fastest way to recover from breaking changes.

% \textbf{Context:}
% \textbf{Objective:}
% \textbf{Method:}
% \textbf{Results:}
% \textbf{Conclusions:}

% Palavras-chaves em inglês, separadas por ponto.
\keywords{npm. Breaking change. Semantic Version. Dependencies}
\end{abstract}