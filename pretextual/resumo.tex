\begin{resumo}
%Elemento obrigatório, constituído de uma sequência de frases concisas e objetivas, em forma de texto.  Deve apresentar os objetivos, métodos empregados, resultados e conclusões.  O resumo deve ser redigido em parágrafo único, conter no máximo 500 palavras e ser seguido dos termos representativos do conteúdo do trabalho (palavras-chave).

\textbf{Contexto:} Os pacotes hospedados no \textsf{npm} criam uma rede de dependências, na qual o pacote \textit{cliente} depende do pacote \textit{provedor}. Por vezes, os provedores introduzem \textit{breaking changes}, que são alterações que podem causar defeitos nos clientes. Essas alterações deveriam ser publicadas apenas no nível \textit{major} do Versionamento Semântico, mas quando são introduzidas nos níveis \textit{minor} ou \textit{patch}, podem causar defeitos inesperados nos clientes.
% Objetivo
\textbf{Objetivo:} Este trabalho propõe um estudo sobre \textit{breaking changes} em níveis \textit{minor} e \textit{patch} no \textsf{npm}. Os objetivos são: 1) mensurar e 2) categorizar as \textit{breaking changes} e 3) analisar como os clientes se recuperam.
% Método
\textbf{Método:} De uma amostra de clientes do \textsf{npm} foram restauradas as \textit{releases} e instalada a última versão dos provedores que o cliente aceitava no momento da \textit{release}. Em seguida, foram executados os \textit{scripts} \texttt{npm install/test}. Para todas as \textit{releases} que resultaram em erros, foram analisados os códigos e os repositórios dos clientes e dos provedores para verificar se o erro foi causado por um provedor, ou seja, uma \textit{breaking change}.
% Resultados
\textbf{Resultados:} Ao todo, 55 provedores introduziram \textit{breaking changes} que impactaram 13.9\% das \textit{releases} dos clientes e essas \textit{breaking changes} cresceram 63.4\% comparadas ao respectivo ano anterior. Ainda, 54.9\% das \textit{releases} dos provedores com \textit{breaking changes} possuem mais \textit{commits} que as suas demais \textit{releases}. As \textit{breaking changes} são introduzidas nos níveis \textit{minor} e \textit{patch} no mesmo percentual, mas a maioria é corrigida pelos provedores em níveis \textit{patch} e são documentadas em 78.1\% dos casos, principalmente em \textit{issues}, o que faz com que a correção ocorra 3.3 vezes mais rápida. Enquanto os provedores indiretos são os que mais introduzem as \textit{breaking changes}, os clientes as corrigem em 39.1\% dos casos e preferem realizar um \textit{upgrade} na versão do provedor, mas sem alterar o \textit{range}.
% Conclusões
\textbf{Conclusões:} As \textit{breaking changes} realmente ocorrem em \textit{releases minor} e \textit{patch}. Enquanto estudos anteriores focaram em alterações de \textit{APIs}, esse estudo utilizou os testes dos clientes para encontrar qualquer tipo de \textit{breaking change}. Finalmente, apresentamos várias sugestões para os desenvolvedores melhorarem suas interações com o ecossistema do \textsf{npm}.

\palavraschaves{\textsf{npm}. \textit{Breaking change}. Versionamento Semântico. Gerenciamento de dependências.}
\end{resumo}