\begin{resumo}
%Elemento obrigatório, constituído de uma sequência de frases concisas e objetivas, em forma de texto.  Deve apresentar os objetivos, métodos empregados, resultados e conclusões.  O resumo deve ser redigido em parágrafo único, conter no máximo 500 palavras e ser seguido dos termos representativos do conteúdo do trabalho (palavras-chave).


TEXTO TEXTO TEXTO TEXTO TEXTO TEXTO TEXTO TEXTO TEXTO TEXTO TEXTO TEXTO TEXTO TEXTO TEXTO TEXTO TEXTO TEXTO TEXTO TEXTO TEXTO TEXTO TEXTO TEXTO TEXTO TEXTO TEXTO TEXTO TEXTO TEXTO TEXTO TEXTO TEXTO TEXTO TEXTO TEXTO TEXTO TEXTO TEXTO TEXTO TEXTO TEXTO TEXTO TEXTO TEXTO TEXTO TEXTO TEXTO TEXTO TEXTO TEXTO TEXTO TEXTO TEXTO TEXTO TEXTO TEXTO TEXTO TEXTO TEXTO TEXTO TEXTO TEXTO TEXTO TEXTO TEXTO TEXTO TEXTO TEXTO TEXTO TEXTO TEXTO TEXTO TEXTO TEXTO TEXTO TEXTO TEXTO TEXTO TEXTO TEXTO TEXTO TEXTO TEXTO TEXTO TEXTO TEXTO TEXTO TEXTO TEXTO TEXTO TEXTO TEXTO TEXTO TEXTO TEXTO TEXTO TEXTO TEXTO TEXTO TEXTO TEXTO TEXTO TEXTO TEXTO TEXTO TEXTO TEXTO TEXTO TEXTO TEXTO TEXTO TEXTO TEXTO TEXTO TEXTO TEXTO TEXTO TEXTO TEXTO TEXTO TEXTO TEXTO TEXTO TEXTO TEXTO TEXTO TEXTO TEXTO TEXTO TEXTO TEXTO

% TODO: se possível, escreva um resumo estruturado. Para TCC 1, o resumo estruturado teria os seguintes elementos:
% \textbf{Contexto:} \\
% \textbf{Objetivo:} \\
% \textbf{Método:} \\
% \textbf{Resultados esperados:} 
% ou, para TCC 2:
% \textbf{Contexto:} \\
% \textbf{Objetivo:} \\
% \textbf{Método:} \\
% \textbf{Resultados:} \\
% \textbf{Conclusões:}

% Palavras-chaves, separadas por ponto (tente não definir mais do que cinco)
\palavraschaves{}
\end{resumo}