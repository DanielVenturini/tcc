\begin{resumo}
%Elemento obrigatório, constituído de uma sequência de frases concisas e objetivas, em forma de texto.  Deve apresentar os objetivos, métodos empregados, resultados e conclusões.  O resumo deve ser redigido em parágrafo único, conter no máximo 500 palavras e ser seguido dos termos representativos do conteúdo do trabalho (palavras-chave).

% TODO: se possível, escreva um resumo estruturado. Para TCC 1, o resumo estruturado teria os seguintes elementos:
\textbf{Contexto:} O \textsf{npm} é o gerenciador de pacotes para linguagem de programação \textsf{Javascript}. Os pacotes hospedados no \textsf{npm} dependem um dos outros, criando uma rede de dependências entre eles, no qual o cliente é o pacote que depende de um outro pacote, enquanto que o provedor é o pacote que fornece recursos aos seus dependentes. Entretanto, os provedores evoluem independentemente dos seus clientes e, por vezes, introduzem alterações que podem causar defeitos nos clientes. Essas alterações são chamadas de \textit{breaking changes} e podem se tornar um problema para os clientes.\\
\textbf{Objetivo:} Este trabalho propõe: 1) mensurar a ocorrência de \textit{breaking changes} no \textit{npm}; 2) categorizar as \textit{breaking changes}; e 3) analisar como os clientes realizam modificações em seu código para reagir a uma \textit{breaking change}.\\
\textbf{Método:} De uma amostra dos clientes do \textit{npm}, restaurar seus arquivos para uma determinada \textit{release} e instalar os provedores na última versão que o cliente aceitava no momento da \textit{release}. Após, executar os clientes através dos comandos \textit{npm install/test}. Todas as \textit{releases} que resultaram em erros serão analisadas para verificar se o erro foi causado por um provedor, sendo assim uma \textit{breaking change}. Então, serão recuperadas informações nos repositórios dos provedores para categorizar as \textit{breaking changes} e nos repositórios dos clientes para analisar como os clientes reagiram às \textit{breaking changes}.\\
% \textbf{Resultados esperados:} 
% ou, para TCC 2:
% \textbf{Contexto:} \\
% \textbf{Objetivo:} \\
% \textbf{Método:} \\
% \textbf{Resultados:} \\
% \textbf{Conclusões:}

% Palavras-chaves, separadas por ponto (tente não definir mais do que cinco)
\palavraschaves{\textsf{npm}. \textit{Breaking change}. Versionamento Semântico. Dependências}
\end{resumo}