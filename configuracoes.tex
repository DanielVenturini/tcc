%
% Esse arquivo conterá pacotes e comandos utilizados na monografia
%
% Observação - devido a um erro do sharelatex foi necessário colocar na raiz do projeto os seguintes arquivos:
% gcnumparser.sty, fcprefix.sty, fmtcount.sty, fc-poruges.def, fcportuguese.def.
% Tal problema foi relatado em: https://github.com/nlct/fmtcount/issues/26
% Quando o sharelatex corrigir o problema acredito que podemos remover esses arquivos do projeto. At. Luiz Arthur.
%

% Este comando não é necessário: utilizei apenas para deixar o latex2rtf
% feliz (e descobrir a codificação do texto).
\usepackage[utf8]{inputenc}

% Suporte a figuras e subfiguras
\usepackage{graphics}
\usepackage{subfigure}

% Suporte a tabelas (principalmente do cronograma)
\usepackage{tabularx}
\usepackage{multirow}
\usepackage{array}
\usepackage{tabularx}
\usepackage{colortbl}
\usepackage{hhline}
\usepackage{xcolor}


% better tables
\usepackage{booktabs}

% frame box
\usepackage{mdframed}

% Escalar fontes para redimencionar, por exemplo tabelas
\usepackage{scalefnt}

% Algoritmos.
\usepackage{algorithm,algorithmic}

\usepackage[alf]{abntex2cite}

% the table's styles
\usepackage{booktabs}

% Elementos geralmente utilizados na tabela do cronograma
\newcommand{\fullcell}{\multicolumn{1}{>{\columncolor[gray]{0.5}}c}{}}
\newcommand{\fullcellline}{\multicolumn{1}{>{\columncolor[gray]{0.5}}c|}{}}
\newcommand{\mc}[3]{\multicolumn{#1}{#2}{#3}}
\newcommand{\y}{\rule{8pt}{4pt}}
\newcommand{\n}{\hspace*{8pt}}

% Define o caminho das figuras
\graphicspath{{images/}}

%% Configuração de glossário
\usepackage[portuguese]{nomencl}
\usepackage[nogroupskip,acronym,nomain,nonumberlist,nopostdot,nohypertypes={acronym}]{glossaries}

\makenoidxglossaries

% para siglas em português
\newcommand{\sigla}[2]
{
 \newglossaryentry{#1}{
  name=#1,
  description={#2},
  first={#2 (#1)},
  long={#2}
 }  
}

% para siglas de língua estrangeira, nessas a descrição longa fica em itálico.
\newcommand{\siglaIt}[2]
{
 \newglossaryentry{#1}{
  name=#1,
  description={\textit{#2}},
  first={\textit{#2} ({#1})},
  long={\textit{#2}}
 }  
}

% --- Estilos para apresentação de Código ----- %
\usepackage{listings}
\lstset{escapechar=§}
\lstloadaspects{formats}

\lstset{
	aboveskip=0cm,
	stringstyle=\ttfamily,
	showstringspaces = false,
	basicstyle=\scriptsize\ttfamily,
	commentstyle=\color{gray!45},
	keywordstyle=\bfseries,
	ndkeywordstyle=\bfseries,
	identifierstyle=\ttfamily,
	numbers=left,
	numbersep=15pt,
	numberstyle=\tiny,
	numberfirstline = false,
	breaklines=true
}

\lstdefinelanguage{JavaScript}{
	keywords={typeof, new, true, false, catch, function, return, null, catch, switch, var, const, let, async, await, if, in, while, do, else, case, break, from},
	ndkeywords={class, export, boolean, throw, implements, import, this},
	sensitive=false,
	comment=[l]{//},
	morecomment=[s]{/*}{*/},
	morestring=[b]',
	morestring=[b]"
}

% Diff language
\usepackage{xcolor}
\definecolor{diffstart}{named}{lightgray}
\definecolor{diffincl}{named}{blue}
\definecolor{diffrem}{named}{red}

\lstset{
	aboveskip=0cm,
	stringstyle=\scriptsize,
	showstringspaces = false,
	basicstyle=\scriptsize\ttfamily,
	commentstyle=\color{gray!45},
	keywordstyle=\bfseries,
	ndkeywordstyle=\bfseries,
	identifierstyle=\ttfamily,
	numbers=left,
	numbersep=15pt,
	numberstyle=\tiny,
	numberfirstline = false,
	breaklines=true
}

\lstdefinelanguage{diff}{
    basicstyle=\scriptsize\ttfamily,
	morecomment=[f][\color{diffstart}]{@@},
	morecomment=[f][\color{diffincl}]{+\ },
	morecomment=[f][\color{diffrem}]{-\ },
	morestring=[b]',
	morestring=[b]",
	keywords={typeof, new, true, false, catch, function, return, null, catch, switch, var, const, let, async, await, if, in, while, do, else, case, break, from},		
	ndkeywords={class, export, boolean, throw, implements, import, this},
}
% --- Fim da Definição de Estilos para apresentação de Código ----- %