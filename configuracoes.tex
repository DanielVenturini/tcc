%
% Esse arquivo conterá pacotes e comandos utilizados na monografia
%
% Observação - devido a um erro do sharelatex foi necessário colocar na raiz do projeto os seguintes arquivos:
% gcnumparser.sty, fcprefix.sty, fmtcount.sty, fc-poruges.def, fcportuguese.def.
% Tal problema foi relatado em: https://github.com/nlct/fmtcount/issues/26
% Quando o sharelatex corrigir o problema acredito que podemos remover esses arquivos do projeto. At. Luiz Arthur.
%

% Este comando não é necessário: utilizei apenas para deixar o latex2rtf
% feliz (e descobrir a codificação do texto).
\usepackage[utf8]{inputenc}

% Suporte a figuras e subfiguras
\usepackage{graphics}
\usepackage{subfigure}

% Suporte a tabelas (principalmente do cronograma)
\usepackage{tabularx}
\usepackage{multirow}
\usepackage{array}
\usepackage{tabularx}
\usepackage{colortbl}
\usepackage{hhline}
\usepackage{xcolor}

% Escalar fontes para redimencionar, por exemplo tabelas
\usepackage{scalefnt}

% Algoritmos.
\usepackage{algorithm,algorithmic}

\usepackage[alf]{abntex2cite}

% Elementos geralmente utilizados na tabela do cronograma
\newcommand{\fullcell}{\multicolumn{1}{>{\columncolor[gray]{0.5}}c}{}}
\newcommand{\fullcellline}{\multicolumn{1}{>{\columncolor[gray]{0.5}}c|}{}}
\newcommand{\mc}[3]{\multicolumn{#1}{#2}{#3}}
\newcommand{\y}{\rule{8pt}{4pt}}
\newcommand{\n}{\hspace*{8pt}}

% Define o caminho das figuras
\graphicspath{{images/}}

%% Configuração de glossário
\usepackage[portuguese]{nomencl}
\usepackage[nogroupskip,acronym,nomain,nonumberlist,nopostdot,nohypertypes={acronym}]{glossaries}

\makenoidxglossaries

% para siglas em português
\newcommand{\sigla}[2]
{
 \newglossaryentry{#1}{
  name=#1,
  description={#2},
  first={#2 (#1)},
  long={#2}
 }  
}

% para siglas de língua estrangeira, nessas a descrição longa fica em itálico.
\newcommand{\siglaIt}[2]
{
 \newglossaryentry{#1}{
  name=#1,
  description={\textit{#2}},
  first={\textit{#2} ({#1})},
  long={\textit{#2}}
 }  
}

% --- Estilos para apresentação de Código ----- %
\usepackage{listings}
\lstloadaspects{formats}

% Opções de listing usados para o código fonte
% Ref: http://en.wikibooks.org/wiki/LaTeX/Packages/Listings

\lstset{ %
language=Java,                  % choose the language of the code
basicstyle=\footnotesize,       % the size of the fonts that are used for the code
%basicstyle=\ttfamily,
stringstyle=\ttfamily\color[rgb]{0.16,0.16,0.16},
numbers=left,                   % where to put the line-numbers
numberstyle=\footnotesize,      % the size of the fonts that are used for the line-numbers
stepnumber=1,                   % the step between two line-numbers. If it's 1 each line will be numbered
numbersep=4pt,                  % how far the line-numbers are from the code
showspaces=false,               % show spaces adding particular underscores
showstringspaces=false,         % underline spaces within strings
showtabs=false,                 % show tabs within strings adding particular underscores
frame=single,	                % adds a frame around the code
framerule=0.6pt,
tabsize=2,	                % sets default tabsize to 2 spaces
captionpos=b,                   % sets the caption-position to bottom
breaklines=true,                % sets automatic line breaking
breakatwhitespace=false,        % sets if automatic breaks should only happen at whitespace
escapeinside={\%*}{*)},         % if you want to add a comment within your code
backgroundcolor=\color[rgb]{1.0,1.0,1.0}, % choose the background color.
rulecolor=\color[rgb]{0.8,0.8,0.8},
extendedchars=true,
xleftmargin=10pt,
xrightmargin=10pt,
framexleftmargin=10pt,
framexrightmargin=10pt
}

\definecolor{javared}{rgb}{0.6,0,0} % for strings
\definecolor{javagreen}{rgb}{0.25,0.5,0.35} % comments
\definecolor{javapurple}{rgb}{0.5,0,0.35} % keywords
\definecolor{javadocblue}{rgb}{0.25,0.35,0.75} % javadoc

\definecolor{DarkBlue}{rgb}{0,0,0.61}
\definecolor{DarkGreen}{rgb}{0,0.4,0}

% Numeros.
\lstdefinestyle{mynumbers}{
	numbers=left,
	stepnumber=1,
	numbersep=4pt,
	numberstyle=\tiny\color{black}
}
% Text Code.
\lstdefinestyle{mytextcode}{
	basicstyle=\footnotesize,
	tabsize=2,
	showspaces=false,
	showstringspaces=false,
	extendedchars=true,
	breaklines=true
}
% Frame.
\lstdefinestyle{myframe}{
	backgroundcolor=\color{white},
	frame=trbl
}
% C++ Style.
\lstdefinestyle{C++}{
	language=C++,
	style=mynumbers,
	style=mytextcode,
	style=myframe,
  keywordstyle=\color{black}\bfseries,
  stringstyle=\color{gray},
  commentstyle=\color[rgb]{0.08,0.08,0.08},
  morecomment=[s][\color{lightgray}]{/*}{*/},
  otherkeywords={\#include, \#define, \#pragma, \#typedef, dim3},
  emph={ __device__, __global__, __shared__, __host__, __constant__},
  emphstyle=\color{DarkBlue}\bfseries,
  emph={[2] printf, scanf},
  emphstyle=[2]\color{DarkGreen},
}
% C Style.
\lstdefinestyle{C}{
	language=C,
	style=mynumbers,
	style=mytextcode,
	style=myframe,
	keywordstyle=\color{black}\bfseries,
  	stringstyle=\color{gray},
  	commentstyle=\color[rgb]{0.08,0.08,0.08},
  	morecomment=[s][\color{lightgray}]{/*}{*/},
  	otherkeywords={\#include, \#define, \#pragma, \#typedef, dim3, bool},
  	emph={ __device__, __global__, __shared__, __host__, __constant__},
  	emphstyle=\color{DarkBlue}\bfseries,
  	emph={[2] printf, scanf},
  	emphstyle=[2]\color{DarkGreen},
	backgroundcolor={}
}
% Javascript Style
\lstdefinestyle{Javascript}{
  keywords={typeof, new, true, false, catch, function, return, null, catch, switch, var, if, in, while, do, else, case, break, const},
  keywordstyle=\color{green}\bfseries,
  ndkeywords={class, export, boolean, throw, implements, import, this},
  ndkeywordstyle=\color{darkgray}\bfseries,
  identifierstyle=\color{black},
  sensitive=false,
  comment=[l]{//},
  morecomment=[s]{/*}{*/},
  commentstyle=\color{purple}\ttfamily,
  stringstyle=\color{red}\ttfamily,
  morestring=[b]',
  morestring=[b]"
}
\lstset{
  language=Javascript,
  tabsize=2,
  literate={{=}{{\textcolor{blue}{=}}}1}
}
% Bash Style.
\lstdefinestyle{bash}{
	language=bash,
	style=mynumbers,
	style=mytextcode,
	style=myframe,
	backgroundcolor={},
	frame=single,
	basicstyle=\scriptsize\ttfamily
}
% Python Style.
\lstdefinestyle{python}{
	language=python,
	style=mynumbers,
	style=mytextcode,
	style=myframe,
	backgroundcolor={}
}
% Java Style.
\lstdefinestyle{java}{
	language=java,
	style=mynumbers,
	style=mytextcode,
	style=myframe,
	backgroundcolor={}
}
% ASM Style.
\lstdefinestyle{asm}{
  %belowcaptionskip=1\baselineskip,
  %xleftmargin=\parindent,
  language=[x86masm]Assembler,
  style=mynumbers,
  style=mytextcode,
  style=myframe,
  backgroundcolor={},
  frame=single,
  basicstyle=\scriptsize\ttfamily,
  commentstyle=\itshape\color{purple!40!black},
}

% Fortran Style.
\lstdefinestyle{fortran}{
  language=[90]Fortran,
  style=mynumbers,
  style=mytextcode,
  style=myframe,
  backgroundcolor={},
  frame=single,
  basicstyle=\footnotesize,
  commentstyle=\itshape\color{purple!40!black},
  morecomment=[l]{!\ }% Comment only with space after !
}

% LLVM Style.
\lstdefinestyle{llvm}{
	language=llvm,
	%inputencoding=utf8,
	style=mynumbers,
	style=mytextcode,
	style=myframe,
	backgroundcolor={},
	frame=single,
	basicstyle=\scriptsize\ttfamily,
  tabsize=4,
  %rulecolor=,
  upquote=true,
% aboveskip={1.5\baselineskip},
  columns=fixed,
  prebreak = \raisebox{0ex}[0ex][0ex]{\ensuremath{\hookleftarrow}},
  showtabs=false,
	%basicstyle=\scriptsize\upshape\ttfamily,
  identifierstyle=\ttfamily,
  keywordstyle=\ttfamily\bfseries\color[rgb]{0,0,0},
  %commentstyle=\ttfamily\color[rgb]{0.133,0.545,0.133},
  commentstyle=\ttfamily\color[rgb]{0.08,0.08,0.08},
  %stringstyle=\ttfamily\color[rgb]{0.627,0.126,0.941}
  stringstyle=\ttfamily\color[rgb]{0.16,0.16,0.16}
}

\lstdefineformat{C}{%
	\{=\newline\string\newline\indent,%
	\}=[;]\newline\noindent\string\newline,%
	\};=\newline\noindent\string\newline,%
	;=[\ ]\string\space}

% --- Fim da Definição de Estilos para apresentação de Código ----- %